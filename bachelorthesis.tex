	%% LLT: Turn off some annoying warnings...
\RequirePackage{silence}
\WarningFilter{titlesec}{Non standard sectioning command}
\WarningFilter{scrreprt}{Usage of package}
\WarningFilter{scrreprt}{Activating an ugly workaround}
% **************************************************
% Document Class Definition
% **************************************************
\documentclass[					%
	paper=A4,					% paper size --> A4 is default in Germany
%	twoside=true,				% onesite or twoside printing
    oneside=true,
%	twoside=false,				% onesite or twoside printing
	openright,					% doublepage cleaning ends up right side
	parskip=full,				% spacing value / method for paragraphs
	chapterprefix=true,			% prefix for chapter marks
	10pt,						% font size
	headings=small,				% size of headings		
%	bibliography=totoc,			% include bib in toc
%	listof=totoc,				% include listof entries in toc
	titlepage=on,				% own page for each title page
	captions=tableabove,			% display table captions above the float env
	draft=false, 				% value for draft version
]{scrreprt}						%

\usepackage{csvsimple,longtable,booktabs}
\usepackage{tocloft}
\usepackage{graphicx}				% Grafiken
\usepackage[utf8]{inputenc}			% defines file's character encoding
\usepackage{paralist}          		% for compactitem itemization
\usepackage{chngcntr}           		% Einfache Nummerierung von Abbildungen
\counterwithout{figure}{chapter} 	% Einfache Nummerierung von Abbildungen
\usepackage{listings} 				% Code Listings
\usepackage{float} 					% exaktes Figure Placement mit H
\usepackage[center]{caption}          		% Figure-Captions formatieren
\usepackage[table,xcdraw]{xcolor}
\usepackage[german,ruled]{algorithm2e}

% change font-size algorithm2e
\SetAlFnt{\footnotesize}
\SetAlgoLined

\lstset{
  breaklines=true,
%  numbers=left,
  numbersep=5pt,
  basicstyle=\footnotesize\ttfamily,
  numberstyle=\tiny\color{black}
%  literate=%
%  {Ö}{{\"O}}1
%  {Ä}{{\"A}}1
%  {Ü}{{\"U}}1
%  {ß}{{\ss}}2
%  {ü}{{\"u}}1
%  {ä}{{\"a}}1
%  {ö}{{\"o}}1
}
\lstset{literate=
  {á}{{\'a}}1 {é}{{\'e}}1 {í}{{\'i}}1 {ó}{{\'o}}1 {ú}{{\'u}}1
  {Á}{{\'A}}1 {É}{{\'E}}1 {Í}{{\'I}}1 {Ó}{{\'O}}1 {Ú}{{\'U}}1
  {à}{{\`a}}1 {è}{{\`e}}1 {ì}{{\`i}}1 {ò}{{\`o}}1 {ù}{{\`u}}1
  {À}{{\`A}}1 {È}{{\'E}}1 {Ì}{{\`I}}1 {Ò}{{\`O}}1 {Ù}{{\`U}}1
  {ä}{{\"a}}1 {ë}{{\"e}}1 {ï}{{\"i}}1 {ö}{{\"o}}1 {ü}{{\"u}}1
  {Ä}{{\"A}}1 {Ë}{{\"E}}1 {Ï}{{\"I}}1 {Ö}{{\"O}}1 {Ü}{{\"U}}1
  {â}{{\^a}}1 {ê}{{\^e}}1 {î}{{\^i}}1 {ô}{{\^o}}1 {û}{{\^u}}1
  {Â}{{\^A}}1 {Ê}{{\^E}}1 {Î}{{\^I}}1 {Ô}{{\^O}}1 {Û}{{\^U}}1
  {œ}{{\oe}}1 {Œ}{{\OE}}1 {æ}{{\ae}}1 {Æ}{{\AE}}1 {ß}{{\ss}}1
  {ű}{{\H{u}}}1 {Ű}{{\H{U}}}1 {ő}{{\H{o}}}1 {Ő}{{\H{O}}}1
  {ç}{{\c c}}1 {Ç}{{\c C}}1 {ø}{{\o}}1 {å}{{\r a}}1 {Å}{{\r A}}1
  {€}{{\EUR}}1 {£}{{\pounds}}1
}

\usepackage{minted} % needed for the inclusion of source code
\usepackage{pgfplots} % needed for drawing diagrams
\usepackage{textcomp}
\usepackage{amsmath}
\usepackage{amssymb}
\usepackage{mathtools} 
\usepackage{array}
\usepackage{color}
\usepackage{colortbl}
\usepackage{multirow}
\usepackage{hhline}
% **************************************************
% Information and Commands for Reuse
% **************************************************
\newcommand{\thesisTitle}{Bachelorarbeit}
\newcommand{\thesisName}{Lukas Abegg}
\newcommand{\thesisMatrikelNr}{Matrikelnummer 798972}
\newcommand{\thesisSubtitle}{Suchoptimierung mittels maschinellen Lernens}
\newcommand{\thesisZeitraum}{Zeitraum 04.07.2016 - 11.10.2016}
\newcommand{\thesisSemester}{Sommersemester 2016}
\newcommand{\thesisDate}{Oktober 11, 2016}

\newcommand{\thesisFirstReviewer}{Prof. Dr. habil. Alexander Löser}
\newcommand{\thesisFirstPosition}{Fachbereich VI - Informatik und Medien}
\newcommand{\thesisFirstReviewerUniversity}{\protect{Beuth Hochschule f{\"u}r Technik}}
\newcommand{\thesisFirstReviewerUniversitySmall}{\protect{Beuth Hochschule}}
\newcommand{\thesisFirstReviewerUniversityCity}{Berlin}
\newcommand{\thesisFirstReviewerUniversityStreetAddress}{Luxemburger Str. 10}
\newcommand{\thesisFirstReviewerUniversityPostalCode}{13353}

\newcommand{\thesisSecondReviewer}{Prof. Dr. Martin Oellrich}
\newcommand{\thesisSecondPosition}{Fachbereich II - Mathematik - Physik - Chemie}
\newcommand{\thesisSecondReviewerUniversity}{\protect{Beuth Hochschule f{\"u}r Technik}}
\newcommand{\thesisSecondReviewerUniversitySmall}{\protect{Beuth Hochschule}}
\newcommand{\thesisSecondReviewerUniversityCity}{Berlin}
\newcommand{\thesisSecondReviewerUniversityStreetAddress}{Luxemburger Str. 10}
\newcommand{\thesisSecondReviewerUniversityPostalCode}{13353}

\newcommand{\thesisUniversity}{\protect{Beuth Hochschule f{\"u}r Technik}}
\newcommand{\thesisUniversityDepartment}{Studiengang Medieninformatik (B.Sc.)}
\newcommand{\thesisFachsemester}{Fachsemester 6}
\newcommand{\thesisUniversityCity}{Berlin}
\newcommand{\thesisUniversityStreetAddress}{Luxemburger Str. 10}
\newcommand{\thesisUniversityPostalCode}{13353}

% **************************************************
% Debug LaTeX Information
% **************************************************
%\listfiles

% **************************************************
% Load and Configure Packages
% **************************************************
\usepackage[german]{babel}    		% Deutsche Sprache in automatisch generiertem
\usepackage[							% use bachelorthesis style
	figuresep=colon,%
	sansserif=false,%
	hangfigurecaption=false,%
	hangsection=false,%
	hangsubsection=false,%
	colorize=full,%
	colortheme=bluemagenta,%
	bibsys=bibtex,%
	bibfile=bib-refs,%
	bibstyle=alphabetic,	%
]{bachelorthesis}

\hypersetup{								% setup the hyperref-package options
	pdftitle={\thesisTitle},				% 	- title (PDF meta)
	pdfsubject={\thesisSubtitle},		% 	- subject (PDF meta)
	pdfauthor={\thesisName},				% 	- author (PDF meta)
	plainpages=false,					% 	-
	colorlinks=false,					% 	- colorize links?
	pdfborder={0 0 0},					% 	-
	breaklinks=true,						% 	- allow line break inside links
	bookmarksnumbered=true,				%
	bookmarksopen=true					%
}

	\sloppy

% **************************************************
% Document CONTENT
% **************************************************
\begin{document}

\setstretch{1.1}
 
% --------------------------
% rename document parts
% --------------------------
\renewcaptionname{german}{\figurename}{Abb.}
\renewcaptionname{german}{\tablename}{Tab.}
\renewcommand\listingscaption{Code}
\renewcommand\listoflistingscaption{Sourcecode-Verzeichnis}
\renewcaptionname{german}{\listfigurename}{Abbildungs-Verzeichnis}
\renewcaptionname{german}{\listtablename}{Tabellen-Verzeichnis}
\renewcommand{\cftlottitlefont}{\color{ctcolorblack}\huge \fontfamily{phv}\selectfont}
\renewcommand{\cftloftitlefont}{\color{ctcolorblack}\huge \fontfamily{phv}\selectfont}

% --------------------------
% Formulas
% --------------------------
\setlength{\abovedisplayskip}{0pt}
\setlength{\abovedisplayshortskip}{0pt}
\setlength{\belowdisplayskip}{0pt}
\setlength{\belowdisplayshortskip}{0pt}
    
% --------------------------
% itemize
% --------------------------
\newenvironment{myitemize}
{ \begin{itemize}
    \setlength{\itemsep}{0pt}
    \setlength{\parskip}{0pt}
    \setlength{\parsep}{0pt}     }
{ \end{itemize}                  } 

% --------------------------
% Front matter
% --------------------------
\pagenumbering{roman}				% roman page numbing (invisible for empty page style)
\pagestyle{empty}					% no header or footers
% ------------------------------------  --> cover title page
\begin{titlepage}
	\pdfbookmark[0]{Cover}{Cover}
	\centering
	\hfill
	\vfill
	{\LARGE \color{ctcolortitle}\textbf{\thesisTitle} \\}
	\rule[2pt]{\textwidth}{.4pt} \\
	\Large{\thesisSubtitle} \\
	\small{\thesisZeitraum} \\[25mm]
	
	\begin{center}
	\Large\thesisName\\[5mm]
	\small{\thesisMatrikelNr} \\
	\small{\thesisSemester}\\
	\small{\thesisFachsemester}\\
	\small{\thesisUniversityDepartment} \\
	\small{\thesisUniversity}
	\end{center}		
	\par
	
	\vfill
	\begin{minipage}[t]{.35\textwidth}
		\raggedleft
		\includegraphics[width=4.5cm]{gfx/beuth} \\[2mm]
	\end{minipage}
	\hspace*{15pt}
	\begin{minipage}[t]{.59\textwidth}
		{\Large \thesisFirstReviewerUniversity} \\
	  	{\small \thesisFirstReviewerUniversityStreetAddress} \\[-1mm]
	  	{\small \thesisFirstReviewerUniversityPostalCode\ \thesisFirstReviewerUniversityCity} 
	\end{minipage} \\[15mm]
	\begin{minipage}[t]{.35\textwidth}
		\raggedleft
		\textit{1. Betreuer}
	\end{minipage}
	\hspace*{15pt}
	\begin{minipage}[t]{.59\textwidth}
		\textbf{\thesisFirstReviewer} \\
		{\small \thesisFirstPosition\\
		\thesisFirstReviewerUniversity}
		\vskip 0.2in
	\end{minipage} \\[15mm]
	\begin{minipage}[t]{.35\textwidth}
		\raggedleft
		\textit{2. Betreuer}
	\end{minipage}
	\hspace*{10pt}
	\begin{minipage}[t]{.59\textwidth}
		\textbf{\thesisSecondReviewer} \\
		{\small \thesisSecondPosition\\
		\thesisSecondReviewerUniversity}
	\end{minipage} \\
\end{titlepage}
			% INCLUDE: all titlepages
\pagestyle{plain}					% display just page numbers

\setcounter{tocdepth}{2}				% define depth of toc
\tableofcontents						% display table of contents
\cleardoublepage

% --------------------------
% Body matter
% --------------------------
\pagenumbering{arabic}				% arabic page numbering
\setcounter{page}{1}					% set page counter
\pagestyle{maincontentstyle} 		% fancy header and footer

% !TEX root = ../Bachelorthesis.tex
%
%************************************************
% Einführung
%************************************************
\chapter{Einführung}
\label{sec:Einfuehrung}

Springer Nature ist ein weltweit führender Verlag für Forschungs-, Bildungs- und Fachliteratur mit einer breiten Palette an angesehenen und bekannten Medienmarken und zudem der weltweit größte Verlag für Wissenschaftsbücher. Für das Unternehmen Springer Nature ist es darum wichtig, auf seinen Web-Applikationen eine Suche anbieten zu können, die Suchintentionen erkennt und möglichst schnell zum gesuchten Content leitet. Die Suche wird vor allem als Hilfsmittel zur Navigation und zum Finden von Literatur und Dienstleistungen genutzt. Durch die vielen von Springer Nature publizierten Zeitschriften und Querverweise in Artikeln, wird sie aber auch oft zur Suche nach Issues\footnote{Nummer der Zeitschriftenausgabe, in der sich der Artikel befindet.} und Artikeln verwendet sowie als Hilfestellung um Diagnosen zu Krankheitsbilder stellen zu können.
\\
\\
Springer Nature sammelt viele User-Tracking-Daten und dadurch viel Wissen über das Verhalten der User\footnote{Als User werden die Nutzer der Springermedizin-Suche bezeichnet} bei der Nutzung ihrer Suche, lässt dieses Wissen jedoch bisher noch nicht in ihre Suche einfließen. In dieser Arbeit wollen wir untersuchen, ob mithilfe dieses Wissens, die Suche optimiert werden kann.

\section{Aufbau der Suche bei Springer Nature}
\label{sec:Einfuehrung:AufbauSucheBeiSpringerNature}

\subsubsection{White Label Applikation mit Solr-Suche}
\label{sec:Einfuehrung:AufbauSucheBeiSpringerNature:WhiteLabelApplikationSolr-Suche}

Damit die verschiedenen Verlage und Zeitschriften der Verlagsgruppe Springer Nature ihre Produkte und Dienstleistungen online anbieten können nutzt Springer Nature eine eigens entwickelte White Label Applikation\footnote{Eine White Label Applikation ist eine wiederverwendbare und agil erweiterbare Applikation}. Die White Label Applikation verwendet \textit{Apache Solr} (im Folgenden "Solr" genannt)~(siehe \cite{solr}) als Suchplattform. Die Solr dient hierbei als eine der Schnittstellen zwischen dem Content-Pool von Springer Nature und der White Label Applikation. Bei den vom Content-Pool gelieferten Inhalten, handelt es sich um von Springer Nature Verlag publizierte Zeitschriften, Artikel, Bücher und redaktionelle Inhalte.

\subsubsection{User-Tracking mit Webtrekk}
\label{sec:Einfuehrung:AufbauSucheBeiSpringerNature:Webtrekk}

Um das Verhalten der User auf ihren Web-Applikationen zu tracken verwendet Springer Nature das Analysetool Webtrekk~(siehe \cite{webtrekk}). Die daraus resultierenden Berichte bieten unter anderem die Möglichkeit, \textit{Suchquery-Logs}\footnote{Protokoll über alle ausgeführten Suchanfragen auf der Applikation} und \textit{Click-Trough-Rates} (CTR)\footnote{Kennzahl um die Anzahl der Klicks auf Links im Verhältnis zu den gesamten Impressionen darzustellen} der User auszuwerten.

\pagebreak

\subsubsection{Architektur}
\label{sec:Einfuehrung:AufbauSucheBeiSpringerNature:Architektur}

In Abb. \ref{fig:SucheSpringerNature} ist die Suche nochmals grafisch aufbereitet:

\begin{figure}[H]
\centering
\includegraphics[width=0.5\linewidth]{gfx/AufbauSucheSpringerNature}
\caption[Aufbau der Suche bei Springer Nature]{Aufbau der Suche bei Springer Nature}
\label{fig:SucheSpringerNature}
\end{figure}

\section{Problemstellung: Keine Userrelevanz in der Suche}
\label{sec:Einfuehrung:Problemstellung}

\subsubsection{Userrelevante Dokumente werden nicht gefunden}
\label{sec:Einfuehrung:Problemstellung:Userrelevanz}

Die User von Springermedizin suchen oft mit einschlägig, fundierten Fachbegriffen nach den neuesten und relevantesten Zeitschriften, Bücher oder Publikationen. Die zeitlich aktuellsten Suchtreffer zu finden ist für Springermedizin kein Problem. Die für den User \textit{relevantesten} jedoch schon.

\subsubsection{Der Springer Nature Stakeholder: Springermedizin setzt auf Webtrekk}
\label{sec:Einfuehrung:Problemstellung:Springermedizin}

Zu den Stakeholder\footnote{Bezeichnet Springer Nature interne Kunden, die ein Interesse am Ergebnis der White Label Applikation haben} der in Kapitel \ref{sec:Einfuehrung:AufbauSucheBeiSpringerNature:WhiteLabelApplikationSolr-Suche} angesprochenen White Label Applikation gehört \textit{Springermedizin}~(siehe \cite{SMED}). Springermedizin ist ein Fortbildungs- und Informationsportal für Ärzte. Mithilfe von Webanalysten und Webtrekk versucht Springermedizin das Marketing seines Webauftrittes zu verbessern und ist sehr interessiert an neuen Ansätzen, um die gesammelten Tracking-Daten besser einzusetzen. In dieser Arbeit wird darum der Fokus auf die Verwendung von Tracking-Daten in der Suche von Springermedizin gesetzt. 
 

\subsubsection{Der fast gläserne User}
\label{sec:Einfuehrung:Problemstellung:Glaeserne-User}

Springermedizin sammelt Tracking-Daten über jegliche Aktivitäten auf deren Applikationen und investiert Zeit und Geld in die Individualisierung\footnote{Mit Individualisierung wird die Speicherung eigener Parameter bezeichnet} der Analysedaten auf Webtrekk. Mittlerweile sind knapp 30 Custom-Parameter\footnote{Individuell erzeugte Parameter für Berichte und Analysen} auf Webtrekk angelegt um genau die Daten zu tracken, die zur Analyse des Verhaltens der User auf ihrer Applikationen relevant sind. Dadurch entsteht ein fast \glqq gläsernen User\grqq{}. Dieses Wissen könnte zum Vorteil des Users eingesetzt werden, indem es in der Suche verwendet wird.

\section{Ziel der Arbeit}
\label{sec:Einfuehrung:ZielArbeit}

\subsection{Suchoptimierung durch Click-Trough-Daten}
\label{sec:Einfuehrung:ZielArbeit:Suchoptimierung}

In dieser Arbeit werden wir untersuchen, ob mithilfe der von Springermedizin gesammelten Click-Trough-Daten dessen Suche verbessert werden kann. Im Idealfall widerspiegeln die gesammelten Click-Trough-Daten der Suchresultate die Userrelevanz der einzelnen Dokumente\footnote{Als Dokumente werden die einzelnen Suchresultate bezeichnet}.

\subsubsection{Annahmen}
\label{sec:Einfuehrung:ZielArbeit:Suchoptimierung:Annahmen}

Wir gehen dabei von folgenden Annahmen aus. Relevante Dokumente sind wichtiger als nicht relevante Dokumente. Eine Suchergebnis ist dann gut, wenn die relevanten Ergebnisse in der verwendeten Hierarchie vor den nicht relevanten Ergebnisse auftauchen. 

\subsection{Abbildung auf das Springermedizin-Umfeld}
\label{sec:Einfuehrung:ZielArbeit:AbbildungSpringermedizinUmfeld}

\subsubsection{Potential von Userrelevanzen in der Suchoptimierung analysieren}
\label{sec:Einfuehrung:ZielArbeit:Potential}

Die Analyse von User-Tracking-Daten bietet viel Potential bezogen auf Userrelevanzen. Sind anhand des hier umgesetzten Lösungsansatzes Verbesserungen in der Qualität der Suche zu verzeichnen, möchte Springermedizin in Zukunft vermehrt User-Tracking-Daten in die Suche einfließen lassen. Diese Arbeit könnte dann als Fundament für weitere Lösungsansätze dienen.

\subsubsection{Bekanntes und wirkungsvolles Information Retrieval Verfahren}
\label{sec:Einfuehrung:ZielArbeit:AbbildungSpringermedizinUmfeld:InformationRetrievalVerfahren}

Suchoptimierung mittels Userrelevanz ist ein bekanntes und nicht triviales, aber relativ wirkungsvolles Information Retrieval Verfahren~(siehe \cite{IWUSBI}). Seit Mitte der 2000er Jahre wird mithilfe dieses Verfahrens versucht, Suchmaschinen zu verbessern. Aus dieser Zeit stammen auch die ersten Ansätze um mithilfe von Click-Trough-Daten die Userrelevanz der Suchergebnisse zu berechnen~(siehe \cite{Joachims}).

\subsubsection{Lösungsansatz basierend auf Click-Trough-Daten aus Webtrekk}
\label{sec:Einfuehrung:ZielArbeit:AbbildungSpringermedizinUmfeld:Loesungsansatz}

Springermedizin führt ein eigenes Tracking der User durch und verwendet auf Webtrekk selbst definierte Tracking-Parameter. Dadurch hängt die Wahl des in dieser Arbeit zu untersuchenden Lösungsansatzes und dessen Umsetzung stark von den durch Webtrekk gegeben Analyse-Daten ab.

\subsubsection{Anwendung auf Springermedizin-Umfeld}
\label{sec:Einfuehrung:ZielArbeit:AbbildungSpringermedizinUmfeld:Adaptierung}

Bei der Verwendung von Userrelevanzen in der Suche handelt es sich um ein bekanntes und gut erforschtes Problem. Wir werden in dieser Arbeit versuchen, einen bestehenden Lösungsansatz (Position-based Modell) auf das Springermedizin-Umfeld abzubilden. Die Herausforderung wird hierbei die Adaptierung des Lösungsansatzes auf das Springermedizin-Umfeld sein.

\subsubsection{Was wird in dieser Arbeit nicht behandelt?}
\label{sec:Einfuehrung:ZielArbeit:AbbildungSpringermedizinUmfeld:NichtBehandeln}

Durch den vorgegebenen Zeitraum für die Erstellung dieser Bachelorarbeit bedingt, werden wir den Lösungsansatz so wählen, dass er mit den Gegebenheiten bei Springermedizin sinnvoll und in diesem Zeitrahmen realistisch implementiert werden kann. Wir werden daher in dieser Arbeit keine Gegenüberstellung mit anderen Lösungsansätzen machen. 
\\
\\
Bei der Umsetzung des Lösungsansatzes konzentrieren wir uns auf die Implementation des Algorithmus zur Berechnung der Userrelevanz. Die semantische Aufschlüsselung von Suchtermen ist nicht Kern dieser Arbeit. Die semantische Aufschlüsselung des Suchterms\footnote{Als Suchterm wird eine Suchanfragen bezeichnet} zur Analyse der Webtrekk-Daten enthält darum keine Gewichtungen der Relationen zwischen Webtrekk-Daten und Suchterm. Alle Relationen werden gleich gewichtet. 

\section{Methodik}
\label{sec:Einfuehrung:Methodik}

\subsection{Einführung}
\label{sec:Einfuehrung:Methodik:Einfuehrung}

Wie in Kapitel \ref{sec:Einfuehrung:ZielArbeit:Suchoptimierung} angesprochen, wollen wir das Klick-Verhalten der User in der Suche analysieren um mithilfe der daraus berechenbaren Userrelevanzen die Suchergebnisse zu verbessern. Dieses Klick-Verhalten können wir aus den Click-Trough-Daten lesen. Um mit Click-Trough-Daten arbeiten zu können, müssen wir zuerst verstehen, was Click-Trough-Daten sind und wie sie entstehen. 

\subsection{Click-Trough-Daten verstehen}
\label{sec:Einfuehrung:Methodik:Click-Trough-Daten}

\subsubsection{Was sind Click-Trough-Daten und wie entstehen diese?}
\label{sec:Einfuehrung:Methodik:Click-Trough-Daten:WasSindClick-Trough-Daten}

Click-Trough-Daten sind Tracking-Daten. Tracking-Daten entstehen durch die Interaktion zwischen dem User der Applikation und der Applikation selbst. Sie verfolgen das Verhalten der User auf der Applikation und speichern diese in einer Datenbank, in unserem Fall in Webtrekk ab. Die für uns interessanten Tracking-Daten entstehen, wenn der User auf der Suche von Springermedizin ein Anfrage stellt und darauf folgend, ein Element aus dem Suchresultat anklickt.

\subsubsection{Wie werden die Click-Trough-Daten in Webtrekk gespeichert?}
\label{sec:Einfuehrung:Methodik:Click-Trough-Daten:SpeichernClick-Trough-Daten}

Die Speicherung der Daten auf Webtrekk übernimmt die Springermedizin-Applikation. Führt ein User eine Suche durch und klickt dabei ein Resultat an, sendet die Springermedizin-Applikation die Tracking-Informationen an Webtrekk. Die Tracking-Daten für diese Aktion, setzen sich zusammen aus der Suchanfrage, dem Zeitpunkt der Suche, den Userdaten, der angeklickten Position im Suchresultat und den Dokumentinformationen zum angeklickten Dokument. Aus diesen Daten werden die Click-Trough-Daten erstellt, mithilfe denen wir die Userrelevanz berechnen werden.

\subsubsection{Wie können wir Click-Trough-Daten aus Webtrekk lesen?}
\label{sec:Einfuehrung:Methodik:Click-Trough-Daten:LesenClick-Trough-Daten}

Webtrekk ist ein Analysetool. Das heißt für uns, wir können nicht direkt auf die Datenbank mit den Tracking-Daten zugreifen. Um die Tracking-Daten lesen zu können, müssen wir eine Analyse auf Webtrekk ausführen. Mithilfe dieser Analyse können wir uns die Click-Trough-Daten so zusammenstellen lassen, wie wir sie für die Berechnung der Userrelevanz benötigen.
\\
\\
Die Click-Trough-Daten bestehen aus einzelnen Click-Trough-Rates. Eine Click-Trough-Rate zeigt die Anzahl der Klicks, die zu einer bestimmten Suchanfrage auf ein bestimmtes Dokument gemacht wurden und auf welcher Position im Suchresultat sich dieses Dokument dabei befunden hat. Die Webtrekk-Analysen geben uns eine Sammlung von Click-Trough-Rates zurück. Wir können bei diesen Analysen die Click-Trough-Rates nach Suchbegriffen oder auch Suchtermen filtern und den Zeitraum mitgeben, in welchen die Suchanfragen durchgeführt wurden. Des weiteren gibt es die Möglichkeit weitere Filter wie die Anzahl zurückzugebender Click-Trough-Rates oder auch den \glqq Login-Status\footnote{Mit Login-Status wird zwischen einem zum Zeitpunkt der Suche auf der Springermedizin-Applikation angemeldeten und nicht angemeldeten User unterschieden} des Users\grqq{} zu setzen. 

\subsubsection{Wie sehen die Click-Trough-Daten aus?}
\label{sec:Einfuehrung:Methodik:Click-Trough-Daten:AussehenClick-Trough-Daten}

Eine Beispiel für eine Click-Trough-Rate wie sie von einer Webtrekk-Analyse ausgespielt wird, sieht wie folgt aus:

\begin{tabular}{|p{0.8\textwidth}|p{0.15\textwidth}|}\hline
	\textbf{Click-Trough-Rate} & \textbf{Anzahl Klicks} \\ \hline
	searchresult-1.Course.chronische Dyspnoe bei Erwachsenen.10621768.chronische Dyspnoe & 1 \\ \hline
 \end{tabular}

Hier die Aufschlüsselung der Click-Trough-Rate:

\begin{tabular}{|p{0.15\textwidth}|p{0.15\textwidth}|p{0.27\textwidth}|p{0.1\textwidth}|p{0.2\textwidth}|}\hline
	\textbf{Position} & \textbf{Dokumenttyp} & \textbf{Titel} & \textbf{ID} & \textbf{Suchterm} \\ \hline
	searchresult-1 & Course & chronische Dyspnoe bei Erwachsenen & 10621768 & chronische Dyspnoe \\ \hline
 \end{tabular}
 
Die Click-Trough-Rate lässt sich wie folgt lesen. In diesem Beispiel haben die User mit der Suchanfrage \glqq chronische Dyspnoe\footnote{Als Dyspnoe wird eine unangenehm erschwerte Atemtätigkeit bezeichnet}\grqq{} gesucht. Dabei haben sie das Dokument mit der ID 10621768 angeklickt. Dieses hat sich dabei auf der Position 1 der Suchresultate gefunden. Es wurde insgesamt einmal angeklickt in der gesuchten Periode. 

\subsection{Reranking mittels Click-Trough-Rate}
\label{sec:Einfuehrung:Methodik:Reranking}

Im vorherigen Abschnitt haben wir gelernt wie Click-Trough-Daten entstehen und wie sie zu lesen sind. Nun können wir mit diesem Wissen die Userrelevanzen der Dokumente im Suchresultat zur Suchanfrage berechnen. Mithilfe der berechneten Userrelevanzen werden wir dann ein \textit{Reranking}\footnote{Mit Reranking bezeichnen wie die Umsortierung einer Liste von Suchresultaten} der Suchresultate durchführen. So wollen wir die Userrelevanz in die Suche einbinden. Die Vorgehensweise dazu sieht wie folgt aus.

\subsubsection{Suchterm semantisch aufschlüsseln}
\label{sec:Einfuehrung:Methodik:Reranking:SuchtermSegmentierung}

Um die Userrelevanz berechnen zu können müssen wir zunächst die relevanten Click-Trough-Daten filtern. Click-Trough-Daten müssen nicht immer mit dem vollständigen Suchterm in Relation stehen. Sie können auch nur mit einem Wort des Suchterms oder einem Synonym des Wortes in Relation stehen. Wir müssen darum den Suchterm semantisch aufschlüsseln um alle relevanten Click-Trough-Daten filtern zu können. 

\subsubsection{Aufbereitung Click-Trough-Daten}
\label{sec:Einfuehrung:Methodik:Reranking:Click-Trough-Daten}

Können wir alle relevanten Click-Trough-Daten zu einer Suchanfrage filtern, müssen lernen wie wir diese richtig aufbereiten, um die Userelevanzen berechnen zu können. 

\subsubsection{Userrelevanz in Suchprozess einbinden}
\label{sec:Einfuehrung:Methodik:Reranking:SucheEinbinden}


\subsubsection{Result-Reranking mittels PBM Algorithmus}
\label{sec:Einfuehrung:Methodik:Reranking:Result-RerankingPBM}

\subsubsection{Vergessen der alten Daten}
\label{sec:Einfuehrung:Reranking:Vergessen}


\section{Gliederung und Aufbau}
\label{sec:Einfuehrung:GliederungAufbau}

\subsubsection{Der Lösungsansatz und deren Grundlagen}
\label{sec:Einfuehrung:GliederungAufbau:Loesungsansatz}

Im ersten Kapitel wurde der zu untersuchenden Lösungsansatz vorgestellt. Dabei sind wir auf die Hintergründe dieser Arbeit und die Vorgehensweise eingegangen. Im zweiten Kapitel (Grundlagen) folgt die Theorie des beschriebenen Lösungsansatzes. Hier werden wir uns auf die fachlichen Grundlagen konzentrieren. 

\subsubsection{Umsetzung des Lösungsansatzes}
\label{sec:Einfuehrung:GliederungAufbau:Umsetzung}

In Kapitel 3 (Reranking mittels Click-Trough-Rate Ergebnis) werden wir die in Kapitel \ref{sec:Einfuehrung:Methodik} angesprochene Methodik verfeinern und detailliert die Vorgehensweise bei der Umsetzung diskutieren. Die Umsetzung selbst folgt dann in Kapitel 4 (Implementierung).

\subsubsection{Erkenntnisse verarbeiten}
\label{sec:Einfuehrung:GliederungAufbau:Erkenntnisse}

Um zu prüfen ob der umgesetzte Lösungsansatz die erhofften Verbesserungen erzielt, werden wir diesen in Kapitel 5 (Evaluation und Auswertung) in einer Evaluation mit der bisherigen Springermedizin-Suche vergleichen. Aufgrund der resultierenden Erkenntnisse werden wir in Kapitel 6 ein Fazit ziehen können und einen Ausblick auf mögliche zukünftige Arbeiten geben. 			% INCLUDE: Einführung
% !TEX root = ../thesis-example.tex
%
%************************************************
% Grundlagen
%************************************************
\chapter{Grundlagen}
\label{sec:Grundlagen}

\subsubsection{White Label Applikation mit Solr-Suche}
\label{sec:Grundlagen:AufbauSucheBeiSpringerNature:WhiteLabelApplikationSolr-Suche}


\section{Grundbegriffe}
\label{sec:Grundlagen:Grundbegriffe}

\subsubsection{White Label Applikation mit Solr-Suche}
\label{sec:Grundlagen:AufbauSucheBeiSpringerNature:WhiteLabelApplikationSolr-Suche}
 			% INCLUDE: Grundlagen
% !TEX root = ../thesis-example.tex
%
%************************************************
% Kern der Arbeit
%************************************************
\chapter{Reranking mittels Click-Trough-Rate Ergebnis}
\label{sec:Reranking}

%Prozessaufbau des Lösungsansatzes
%----------------------------------------------------------------

\section{Prozessaufbau des Lösungsansatzes}
\label{sec:Reranking:Prozessaufbau}

\subsection{Prozessaufbau als Bild}
\label{sec:Reranking:Prozessaufbau:ProzessaufbauBild}

\begin{figure}[H]
\centering
\includegraphics[width=0.5\linewidth]{gfx/Prozessaufbau}
\caption[Prozessaufbau des Lösungsansatzes]{Prozessaufbau des Lösungsansatzes}
\label{fig:Prozessaufbau}
\vspace{-2.5em}
\end{figure}

\subsection{Probleme des Lösungsansatzes}
\label{sec:Reranking:Prozessaufbau}

\subsection{Suchterm Segmentierung}
\label{sec:Reranking:Prozessaufbau:SuchtermSegmentierung}

\subsection{Aufbereitung Click-Trough-Daten}
\label{sec:Reranking:Prozessaufbau:Click-Trough-Daten}

\subsubsection{Kein Einfluss auf Suchergebnisqualität während der Klicks}
\label{sec:Reranking:Prozessaufbau:Click-Trough-Daten:Click-Trough-Suchergebnisqualität}

siehe Grundlagen

\subsubsection{Userverhalten}
\label{sec:Reranking:Prozessaufbau:Click-Trough-Daten:Click-Trough-Userverhalten}

Die Webtrekk-Analysen bieten uns nur beschränkte Informationen zum Klick-Verhalten der User. Wichtige Informationen wie die Verweildauer auf einem Dokument oder ob nach diesem Dokument ein weiteres Dokument zum gleichen Suchterm angeklickt worden ist, lassen diese Analysen nicht zu. 

\subsection{Click-Trough-Rate in Suchprozess einbinden}
\label{sec:Reranking:Prozessaufbau:SucheEinbinden}

- Solr gebundene Suchresultatmenge
- Pagination (Einfluss von Reranking)

\subsection{Result-Reranking mittels PBM Algorithmus}
\label{sec:Reranking:Prozessaufbau:Result-RerankingPBM}

- Smoothing

- Mehrfachverwertung von Content 
	=> mehrfache Auflistung in Suchergebnissen (nicht Teil dieser Arbeit)
- Algorithmus nicht festfahren (Overfitting)

- User nicht festfahren auf altem Wissen 
	- aktuelle und neue Publikationen werden nicht berücksichtigt 
		=> keine Userrelevanz (nicht Teil dieser Arbeit)
		
%Methodik
%----------------------------------------------------------------

\section{Methodik}
\label{sec:Reranking:Methodik}

In Kapitel \ref{sec:Grundlagen:Grundbegriffe} haben wir gelernt wie Click-Trough-Daten entstehen und wie sie zu lesen sind. Nun können wir mit diesem Wissen die Click-Trough-Rate der Dokumente berechnen. Mithilfe der berechneten Click-Trough-Daten werden wir dann ein \textit{Reranking} der Suchresultate durchführen. So wollen wir die Userrelevanz in die Suche einbinden. Die Vorgehensweise dazu sieht wie folgt aus.

\subsection{Suchterm Segmentierung}
\label{sec:Reranking:Methodik:SuchtermSegmentierung}

\subsubsection{Suchterm semantisch aufschlüsseln mittels Segmentierung}
\label{sec:Reranking:Methodik:SuchtermSegmentierung:SuchtermSegmentierung}

Um alle relevanten Click-Trough-Daten lesen zu können, müssen wir zunächst den Suchterm auftrennen, wie in Kapitel \ref{sec:Grundlagen:SemantikUserInteraktionen:ProblemstellungenClick-Trough-Daten} angesprochen. Die Auftrennung des Suchterms in die einzelne Worte können wir mithilfe einer Segmentierung\footnote{Bezeichnet die Aufteilung in Abschnitte, in diesem Fall in einzelne Worte} durchführen. Hier könnten wir uns überlegen, zusätzlich mit Stoppwörtern\footnote{Stoppwörter sind Wörter, die sehr häufig auftreten und für gewöhnlich keine Relevanz für den Dokumentinhalt besitzen} nicht relevante Wörter aus dem Suchterm zu entfernen. Dieses Verfahren macht aber im Springermedizin-Kontext keinen Sinn. Wie in Kapitel \ref{sec:Einfuehrung:Problemstellung:Userrelevanz} angesprochen, suchen die User der Springermedizin-Applikation oft mit einschlägig, fundierten Fachbegriffen. Wir gehen darum davon aus, dass alle Wörter des verwendeten Suchterms für das Suchergebnis relevant sind. Diese Erkenntnis basiert auf Aussagen der Redakteure von Springermedizin und Webtrekk-Analysen der meist gesuchtesten Suchtermen der letzten Monate. Auch sind Stoppwörter veraltet und werden in modernen Information Retrieval Verfahren nicht mehr eingesetzt. Wir verzichten darum auf den Einstatz von Stoppwörtern.

\subsubsection{Suchterm semantisch erweitern mittels Thesaurus}
\label{sec:Reranking:Methodik:SuchtermSegmentierung:SuchtermThesaurus}

Wie in Kapitel \ref{sec:Grundlagen:SemantikUserInteraktionen:ProblemstellungenClick-Trough-Daten} thematisiert, wollen wir die Click-Trough-Daten zu unserem Suchterm um Click-Trough-Daten zu verwandten Begriffen erweitern. Für diese semantische Erweiterung eines Suchwortes werden wir einen Thesaurus verwenden. Die Erweiterung umfasst zum Suchterm gleichbedeutende Begriffe (Synonyme), sehr ähnliche Begriffe (Narrow Terms), ähnliche Begriffe im weiteren Sinne (Broader Terms) und verwandte Begriffe (Related Terms).
\\
\\
Springer Nature besitzt einen Webservice mit welchem auf den Thesaurus \textit{Unified Medical Language System} (UMLS)~(siehe \cite{UMLS}) zugegriffen werden kann. Der Webservice nimmt einzelne Wörter und Wort-Listen entgegen. Zu jedem dieser Wörter durchsucht der Webservice den Thesaurus nach den oben erwähnten Arten von verwandten Begriffen. Der Webservice verwendet für diese Suche eine Elasticsearch~(siehe \cite{elasticsearch})\footnote{Eine Elasticsearch ist eine Volltextsuchmaschine}. Die dabei gefundenen Begriffe, liefert der Webservice als Antwort zurück. Um alle relevanten Click-Trough-Daten zu finden, werden wir mit dem segmentierten Suchterm eine Anfrage gegen diesen Webservice stellen und anschließend den segmentierten Suchterm um die gefundenen Begriffe erweitern. Mithilfe des erweiterten Suchterms können wir dann anschließend eine Analyse in Webtrekk starten und alle relevanten Click-Trough-Daten lesen. 

\subsection{Aufbereitung Click-Trough-Daten}
\label{sec:Reranking:Methodik:Click-Trough-Daten}

\subsubsection{Jeder Klick auf ein Dokument ist relevant}
\label{sec:Reranking:Methodik:Click-Trough-Daten:Click-Trough-DatenAuswertungen}

Wie in Kapitel \ref{sec:Reranking:Grundlagen:Click-Trough-Daten:UserverhaltensFeedback} beschrieben, reichen Webtrekk-Analysen für komplexe Auswertungen der Click-Trough-Daten nicht aus. Wir können darum in dieser Arbeit \textit{Feedback-Strategien} für die Click-Trough-Rate Auswertung, wie in \cite{Joachims} beschrieben, nicht verwenden. Stattdessen greifen wir wie ebenfalls in Kapitel \ref{sec:Reranking:Grundlagen:Click-Trough-Daten:UserverhaltensFeedback} beschrieben auf die Click-Trough Features zu, die uns Webtrekk zur Verfügung stellt. Daraus entsteht ein interpretierbares Feature-Set. Dieses ist leider sehr klein und enthält keine Informationen um ein Relevanzfeedback zu den Klick-Häufigkeiten daraus lesen zu können. Wir müssen wir darum davon ausgehen, dass jeder Klick auf ein Dokument relevant ist.

\subsubsection{Gewichtung der Click-Trough-Daten}
\label{sec:Reranking:Methodik:Click-Trough-Daten:Gewichtung}

Durch die semantische Aufschlüsselung des Suchterms haben wir verschieden starke Relationen zwischen Click-Trough-Daten und dem Suchterm. Die Gewichtung der Stärke dieser Relation ist aber nicht Kern dieser Arbeit. Wir gehen darum davon aus, dass unabhängig der stärke der Relation zum Suchterm, alle Click-Trough-Daten eine gleiche Relevanz besitzen.

\subsubsection{Berechnung der Click-Trough-Rate}
\label{sec:Reranking:Methodik:Click-Trough-Daten:Gewichtung}

Die Click-Trough-Rate wird vor allem im Bereich des Internet-Marketing verwendet und stellt grundsätzlich die Anzahl der Klicks auf ein Dokument oder Link im Verhältnis zu den gesamten Impressionen dar. Bezogen auf das in Kapitel \ref{sec:Reranking:Grundlagen:Click-Trough-Daten:UserverhaltensFeedback} angesprochene Feature-Set, würden die \textit{ClickProbability}~(Klick-Wahrscheinlichkeit) direkt als Click-Trough-Rate verwenden. Dazu müssten wir nur die Click-Trough-Daten eines Dokuments ins Verhältnis zu allen Click-Trough-Daten für einer Suchanfrage stellen. Wie wir aber bereits in Kapitel \ref{sec:Grundlagen:SemantikUserInteraktionen} gelernt haben, würden wir damit viele Problemstellungen der Interaktion der User mit der Suche ignorieren. Deswegen verwenden wir den Position-Based Modell basierten Algorithmus um mithilfe dieses angesprochenen Feature-Sets die Click-Trough-Rate zu berechnen.

\subsection{Result-Reranking mittels PBM basiertem Algorithmus}
\label{sec:Reranking:Methodik:Result-RerankingPBM}

\subsubsection{Klick-Wahrscheinlichkeit mit Position-based Modell berechnen}
\label{sec:Reranking:Methodik:Result-RerankingPBM:Klick-Wahrscheinlichkeit}

Wie in Kapitel \ref{sec:Grundlagen:Grundbegriffe:Result-RerankingPBM:AnsatzSucheEinbinden} angesprochen, werden wir unseren Reranking-Algorithmus in die Aufbereitung der Suchresultate aus der Solr-Suche integrieren.
\\
\\
Dieser soll die Suchergebnisliste analysieren, die Click-Trough-Rate der Dokumente berechnen und die Liste neu sortieren. 
\\
\\
Wir müssen jedoch beachten, dass die Solr durch die Pagination-Funktion~(siehe \cite{Pagination}) nur die Top-N-Ergebnisse (bei Springermedizin sind es 20 Ergebnisse) zurückgibt. Dadurch sehen wir nur einen Teil der Suchergebnisse. 
\\
\\
Diese Logik liegt in der Springermedizin-Applikation im Aufbau der Suchanfrage. Daher können wir diese selber steuern und uns statt 20 beispielsweise die nächsten 100 Ergebnisse zurückgeben lassen. Am Ende filtern wir die ersten 20 Ergebnisse und stellen diese dar. Außerdem wissen wir bei diesem Lösungsansatz, in welcher Reihenfolge die Ergebnisse aus der Solr zurückgegeben werden. Wir kennen die Dokumente und deren Rang. Dadurch haben wir hilfreiches Zusatzwissen, welches wir in den Klick-Modell basierten Algorithmus einfließen lassen können.

\subsubsection{Verhältnis zwischen den Klick-Wahrscheinlichkeiten abhängig der Position im Suchresultat definieren}
\label{sec:Reranking:Methodik:Result-RerankingPBM:VerhaeltnisKlick-Wahrscheinlichkeiten}

Aus eigener Erfahrung wissen wir, dass die ersten Dokumente im Suchresultat immer zuerst gesehen werden. Die dahinter gelisteten Dokumente werden fortlaufend analysiert. Dies bestätigt die in Abb. \ref{fig:Grundlage:AnalyseKlicksPositionen} dargestellte Analyse der Klicks auf die ersten 20 Positionen eines Suchergebnisses. Wir sollten darum darauf achten, dass je \textit{schlechter} der Rang des angeklickten Dokumentes im Suchresultat der Solr ist, desto \textit{höher} das Relevanzfeedback zu bewerten ist. 
\\
\\
Das machen wir, indem wir für die Berechnung der Click-Trough-Rate das Verhältnis zwischen Klick-Wahrscheinlichkeit der Position und Klick-Wahrscheinlichkeit des Dokumentes abhängig der Position im Suchresultat definieren. Als Grundlage hierbei dient uns die Position des Suchresultats der Solr. In der folgenden Tabelle sehen wir die Aufteilung der Verhältnisse abhängig der Position:

\begin{table}[H]
\centering
\begin{tabular}{|c|c|c|}\hline
	\textbf{Position} & \textbf{Verhältnis Position zu Dokument}\\ \hline
	1 bis 10 & 1:1 \\ \hline
	11 bis 20 & 1:2 \\ \hline
	größer 20 &  1:3 \\ \hline
 \end{tabular}
 \vspace{.5em}
 \caption[Verhältnis Klick-Wahrscheinlichkeiten der Position zu der des Dokumentes]{Verhältnis Klick-Wahrscheinlichkeiten der Position zu der des Dokumentes}
\label{tab:VerhaeltnisKlick-WahrscheinlichkeitenPositionDokument}
\vspace{-2.5em}
\end{table}

Für die Suchresultate mit einer Position über 20, verstärken wir die Gewichtung der Klick-Wahrscheinlichkeit des Dokumentes erheblich. Das liegt daran, dass bei Klicks auf Dokumente mit einer solch hohen Position wir davon ausgehen können, dass die suchende Person die Suchresultate genau analysiert hat, bevor sie ein Dokument angeklickt hat. Haben wir die Verhältnisse definiert, müssen wir diese in den Algorithmus einbauen. Wie wir dies machen, werden wir im folgenden Abschnitt anschauen.

\subsubsection{Smoothing Faktor in Position-based Modell}
\label{sec:Reranking:Methodik:Result-RerankingPBM:SmoothingPBM}

Wir wissen dass eine Wahrscheinlichkeit einen Wert zwischen 1 und 0 besitzt. Dadurch können Nullwerte entstehen. Das PBM multipliziert die Positions- und Dokument-Wahrscheinlichkeit miteinander, um die Klick-Wahrscheinlichkeit zu berechnen. Wir müssen aber davon ausgehen, dass es Dokumente geben kann, deren Rang nie angeklickt worden ist und umgekehrt. 
\\
\\
Multiplikationen mit Null ergeben immer einen Nullwert. An dieser Stelle führen wir einen \textit{Smoothing-Faktor} ein. Der Smoothing-Faktor soll zwei Probleme lösen. Zum einen wollen wir einen Wahrscheinlichkeitswert trotz der Multiplikation mit Null beachten. Zum anderen wollen wir die im vorherigen Absatz beschriebene Gewichtung abhängig des Relevanzfeedbacks in den Algorithmus einbeziehen. Wir transformieren dazu das Produkt der beiden Wahrscheinlichkeiten in eine gewichtete Summe, dem sogenannten \textit{Weighted Moving Average}~(siehe \cite{weightedAVG}), dessen Gewichte sich zu Eins aufsummieren. Diese Gewichte sind die Smoothing-Faktoren, weshalb das Verfahren zählt zu den Smoothing-Algorithmen zählt.

\subsection{Vergessen der alten Daten}
\label{sec:Reranking:Methodik:Vergessen}

Ein Algorithmus zur Berechnung von Wahrscheinlichkeiten muss sich ein gewisses Grundwissen aneignen. Dies geschieht üblicherweise durch Trainingsdaten. Genauso muss er alte Daten wieder vergessen können, um Overfitting\footnote{Überanpassung des Algorithmus durch zu viele (falsche oder veraltete) Daten} zu vermeiden. 

\subsubsection{Durch Webtrekk ist kein komplexer Lern-Algorithmus notwendig}
\label{sec:Reranking:Methodik:Vergessen:Lern-Algorithmus}

Durch Webtrekk haben wir eine Wissensbasis, die sich stetig und zeitnah aktualisiert. So muss der Algorithmus nicht stetig neues Wissen lernen und altes vergessen, sondern er kann direkt diese Wissensbasis zugreifen. Dies geschieht, indem zur Laufzeit\footnote{Unter Laufzeit wird in diesem Fall der Zeitpunkt der direkte Abfrage während der Suchanfrage bezeichnet} Analysen gegen Webtrekk über eine frei definierbare Periode gemacht werden. Dadurch kann \textit{Overfitting} vermieden werden. Deshalb verwenden wir keinen komplexen Lern-Algorithmen wie in \cite{IWUSBI} vorgestellt.

\subsubsection{Die Klick-Wahrscheinlichkeit ist kein absoluter Wert für die Userrelevanz}
\label{sec:Reranking:Methodik:Vergessen:Relevanzfeedback}

Nun könnten wir die Klick-Wahrscheinlichkeit als absoluten Wert für die \textit{Userrelevanz} betrachten. Dies wäre jedoch falsch, wie in Kapitel \ref{sec:Grundlagen:SemantikUserInteraktionen:RankExamination} analysiert, müssen wir davon ausgehen, dass viele User der Qualität der Suchmaschine vertrauen. Diese betrachten die Top-Suchresultate als die relevanten Suchresultate. Denkbar wäre auch, dass User unabsichtlich das falsche Dokument anklicken und dadurch die Click-Trough-Rate eines Dokumentes verfälschen. Dadurch kann ein \textit{Overfitting} des Algorithmus entstehen.

\subsubsection{Overfitting vermeiden}
\label{sec:Reranking:Methodik:Vergessen:Overfitting}

Um ein Overfitting zu vermeiden, darf der Algorithmus nicht immer anschlagen. Wir müssen sicherstellen, dass vereinzelt zufällige Dokumente in den \glqq Top-Suchresultaten\grqq{} angezeigt werden. So können auch andere Dokumente in den Fokus des Users gerückt werden. Das System fährt sich dadurch nicht auf falschen Annotationen fest. 

\subsubsection{Zusätzliche Varianz durch Zufallsfaktor}
\label{sec:Reranking:Methodik:Vergessen:Zufallsfaktor}

Mithilfe eines Zufallsfaktors kann eine solche Varianz in den Klick-Modell basierten Algorithmus gebracht werden. Wie bereits weiter oben erwähnt, werden viele Suchresultate nie und deren Rang selten bis gar nicht angeklickt. Sie haben darum keine Click-Trough-Daten. Deren Klick-Wahrscheinlichkeit ist entweder Null oder sehr klein. Der Zufallsfaktor soll darum nur leichte Einflüsse in die Klick-Wahrscheinlichkeitsberechnung haben. Auch hier können wir wieder mit dem oben eingeführten \textit{Weighted Moving Average} arbeiten.

%Der PBM-Algorithmus
%----------------------------------------------------------------

\section{Der PBM-Algorithmus}
\label{sec:Reranking:PBM-Algorithmus}


%Zusammenfassung
%----------------------------------------------------------------

\section{Zusammenfassung}
\label{sec:Reranking:Zusammenfassung}

 			% INCLUDE: Kern der Arbeit
% !TEX root = ../thesis-example.tex
%
%************************************************
% Implementierung
%************************************************
\chapter{Implementierung}
\label{sec:Implementierung}

Im letzen Kapitel haben wir unseren Reranking-Algorithmus so detailliert ausgearbeitet, dass er nun implementiert werden kann. In diesem Kapitel geht es nun darum, diesen Algorithmus in der Springermedizin-Suche einzubauen. Mit der Implementierung wollen wir herausfinden, ob der theoretische Ansatz praktisch umgesetzt werden kann und die Gedankengänge bei der Ausarbeitung des Lösungsansatzes korrekt waren. In diesem Kapitel werden wir nicht den Code der Lösung vorstellen. Wir werden aber beschreiben wo wir in die Suche eingreifen, wie wir eingreifen und was wir da genau machen. Das Ziel dieses Kapitels soll es sein, eine Überblick über den implementiert Lösungsansatz zu schaffen.

\section{Architektur der Implementierung}
\label{sec:Implementierung:Architektur}

Um zu verdeutlichen an welcher Stelle des Suchprozesses der Springermedizin-Suche wir eingreifen, sehen wir unten folgend das Prozessbild der Implementierung unseres Lösungsansatzes. Warum wir genau an dieser Stelle eingreifen, haben wir bereits in \ref{sec:Grundlagen:Grundbegriffe:Result-RerankingPBM} ausdiskutiert. Der Suchprozess ist im Prozessbild in mehrere Komponenten aufgeteilt. Die grau hinterlegten Komponenten zeigen bereits bestehende, vom Lösungsansatz unabhängige Teile der Architektur. Die blau hinterlegten Komponenten sind die in der Implementierung neu hinzugefügte Komponenten des Lösungsansatzes. Sie sind in die drei Hauptschritte des Reranking-Algorithmus unterteilt.

\begin{figure}[H]
\centering
\vspace{-1em}
\caption[Prozessbild der Implementierung]{Prozessbild der Implementierung}
\label{fig:ProzessbildImplementierung}
\includegraphics[width=\linewidth]{gfx/ImplementierungProzessbild}
\vspace{-2.5em}
\end{figure}

Wie bereits in Kapitel \ref{sec:Grundlagen:Grundbegriffe:Result-RerankingPBM:AnsatzSucheEinbinden} angesprochen, binden wir den Algorithmus als ein, in sich geschlossenes, unabhängiges Modul zwischen dem Suchprozess und der Aufbereitung des Suchresultats ein. Wie wir in der Abbildung sehen können, wird zuerst der Suchvorgang auf der Solr durchgeführt, bevor wir die Suchergebnisliste entgegennehmen, verarbeiten und mit unserem Reranking-Algorithmus neu sortieren. Die daraus resultierende Ergebnisliste geben dann der Suche als Suchresultat zurück. 

\section{Highlight: PBM basierter Reranking-Algorithmus}
\label{sec:Implementierung:PBM}

Der grafisch dargestellte Prozess in Abb. \ref{fig:ProzessbildImplementierung} entspricht hier nicht nicht der Reihenfolge, wie die Komponenten in der Suche aufgerufen werden, sondern der Reihenfolge der Verarbeitungsschritte im Prozess. In der effektiven Implementierung, übernimmt die Reranking-Komponente die Koordination der Verarbeitungsschritte. Sie wird im Suchprozess direkt \textit{vor der Aufbereitung} der Suchergebnisse aufgerufen und nimmt die Liste der Suchergebnisse der Solr entgegen. Sind alle Schritte des Reranking-Algorithmus verarbeitet, gibt die Reranking-Komponente die \textit{neu sortierte Liste} der Suchergebnisse zurück. Diese wird dann wieder von der Springermedizin-Suche für die Ausgabe als Suchergebnisse aufbereitet.

\subsubsection{Pseudo-Code der Reranking-Komponente}
\label{sec:Implementierung:PBM:Pseudocode}

Um den angesprochenen Vorgang des Rerankings der Suchergebnisse besser zu verstehen, sehen wir hier folgend den Programm-Ablauf der Reranking-Komponente als Pseudo-Code beschrieben:

\begin{figure}[H]
\centering
\vspace{-1em}
\caption[Pseudcode Reranking-Algorithmus]{Pseudcode Reranking-Algorithmus}
\label{fig:PseudcodeRerankingAlgorithmus}
\vspace{.5em}
\DontPrintSemicolon
\begin{algorithm}[H]
\caption{PBM basierter Reranking-Algorithmus}
\BlankLine
\Daten{$searchTerm$ (Suchterm der Suchanfrage), $searchResult$ (zu verarbeitende Suchergebnisliste)}
\Ergebnis{$rerankedSearchResult \leftarrow$ durch Reranking-Algorithmus sortierte Suchergebnisliste}
\BlankLine

\Begin{
	$keywords \leftarrow$ Segmentiere und erweitere Suchterm mittels Thesaurus\;
	$ctrClickDataBySearchTerm \leftarrow$  Lese Click-Through-Daten aus Webtrekk mithilfe von $keywords$\;

	\BlankLine
	\eIf{$ctrClickDataBySearchTerm$ ist gefüllt}{
		\BlankLine
		$ranks \leftarrow$ Lese die angeklickten Positionen aus  $ctrClickDataBySearchTerm$\;
		$ctrClickDataByRanks \leftarrow$ Lese Click-Through-Daten aus Webtrekk mithilfe von $ranks$\;
		
		\BlankLine
		\eIf{$ctrClickDataByRanks$ ist gefüllt}{
			\BlankLine
			\tcc{Berechne Klick-Wahrscheinlichkeit $P(C_{u})$ aller Dokumente $u$}
			\For{ $u \in searchResult$}{
				$\lambda \leftarrow$ Definiere $\lambda$ anhand des vordefinierten Gewichtungsfaktors für die Position $r$ des Dokumentes $u$\;
				$P(E_{r_{u}}) \leftarrow$ Berechne Klick-Wahrscheinlichkeit für Position $r$ des Dokumentes $u$\;
				$P(A_{u}) \leftarrow$ Berechne Klick-Wahrscheinlichkeit für Dokument $u$ zu Suchterm $searchTerm$\;
				\BlankLine
    			$P(C_{u}) \leftarrow \lambda\cdot P(E_{r_{u}}) + (1 - \lambda)\cdot P(A_{u})$\; 
			}
			$ranksByClickProbability \leftarrow$ Sortiere Liste $searchResult$ anhand der $P(C_{u})$ Werte
			
			\BlankLine
			\tcc{Berechne Zufallswert $X_{u}$ aller Dokumente $u$}
			\For{ $u \in searchResult$}{
    			$X_{u} \leftarrow$ Berechne Zufallswert zwischen 1 und $maxPosition(searchResult)$\; 
			}
			$ranksByRandomValue \leftarrow$ Sortiere Liste $searchResult$ anhand der $X_{u}$ Werte
			
			\BlankLine
			\tcc{Berechne effektiven Relevanz-Wert $R_{u}$ aller Dokumente $u$}
			\For{ $u \in searchResult$}{
				$\lambda \leftarrow$ Definiere $\lambda$ anhand des vordefinierten Gewichtungsfaktors für den Zufallswert $X_{u}$\;
				$r_{P(C_{u})} \leftarrow$ Lese Position des Wahrscheinlichkeits-Wertes $P(C_{u})$ aus $ranksByClickProbability$\;
				$r_{X_{u}} \leftarrow$ Lese Position des Zufallswertes $X_{u}$ aus $ranksByRandomValue$\;
				\BlankLine
    			$R_{u} \leftarrow 1 / \left(\lambda\cdot r_{X_{u}} + (1 - \lambda)\cdot r_{P(C_{u})}\right)$\;
			}
			$rerankedSearchResult \leftarrow$ Sortiere Liste $searchResult$ anhand der $R_{u}$ Werte\;
			\tcc{Rückgabe der durch den Reranking-Algorithmus umsortierten Suchergebnisse}
			\KwRet{die ersten 20 Elemente von $rerankedSearchResult$}
		}{	
			\tcc{Keine Click-Through-Daten für die angeklickten Positionen gefunden $\Rightarrow$\\
			Keine Umsortierung der Suchergebnisse mit Reranking-Algorithmus}
			\KwRet{die ersten 20 Elemente von $searchResult$}
		}
	}{
		\tcc{Keine Click-Through-Daten für Suchterm gefunden $\Rightarrow$\\
		Keine Umsortierung der Suchergebnisse mit Reranking-Algorithmus}
		\KwRet{die ersten 20 Elemente von $searchResult$}
	}
}
\end{algorithm}
\end{figure}

\section{Highlight: Webtrekk-Analysen}
\label{sec:Implementierung:Webtrekk}

Wie in der Einführung dieser Arbeit in \ref{sec:Einfuehrung:AufbauSucheBeiSpringerNature:Webtrekk} angesprochen, speichert Springermedizin seine Tracking-Daten auf Webtrekk. Für die Implementierung des Reranking-Algorithmus müssen wir die Click-Trough-Daten für die Wahrscheinlichkeits-Berechnungen wie in \ref{sec:Reranking:Methodik:Result-RerankingPBM} besprochen, mithilfe von Analysen aus Webtrekk abfragen und auswerten. Wie wir diese Analysen abfragen, sehen wir folgend.

\subsubsection{Abfrage von Analysen}
\label{sec:Implementierung:Webtrekk:AnalysenAbfragen}

Der Webservice von Webtrekk bietet verschiedene Schnittstellenmethoden zum Export und Herunterladen der Tracking-Daten an. Die für uns relevante Schnittstellenmethode lautet \textit{getAnalysisData}. Mithilfe dieser Methode können die Daten der verschiedenen Analysen per REST-Schnittstelle\footnote{Representational State Transfer, ist ein Architekturmodell mit dem Webservices mit den Standard-HTTP-Methoden (GET, POST, PUT und DELETE) realisiert werden können} abgerufen werden. Um zu definieren, welche Analyse und mit welchen Filtern diese Analyse abgefragt werden soll, müssen wir folgende Parameter der Methode mitgeben: 

\textbf{Request:} \textit{Springermedizin-Suche $\rightarrow$ Webtrekk REST-Webservice}

\begin{table}[H]
\centering
\vspace{-.75em}
\caption[Beschreibung Analyse-Aufbau Webtrekk-Schnittstelle]{Beschreibung Analyse-Aufbau Webtrekk-Schnittstelle}
\vspace{-.5em}
\label{tab:BeispielCTDaten}
\footnotesize
\renewcommand*{\arraystretch}{1.2}
\resizebox{\textwidth}{!}{%
\begin{tabular}{ll}
\hline
\textbf{Parameter}   & \textbf{Beschreibung}                                                                              \\ \hline
\textit{analysis}    & gewünschte Analyse, alle Analysen der Weboberfläche von Webtrekk sind so abrufbar                  \\
\textit{time\_start} & Startzeit des Analysezeitraumes                                                                    \\
\textit{time\_stop}  & Stopzeit des Analysezeitraumes                                                                     \\
\textit{column}               & Ausgabe bestimmter in der Standard-Analyse nicht vorhandene Datenspalten                           \\
\textit{analysis\_filter}     & Komplexe Filter bestehen aus einem oder mehreren Filtern, die logisch miteinander verknüpft werden \\ \hline
\end{tabular}
}
\vspace{-2em}
\end{table}

Zur Berechnung der Klick-Wahrscheinlichkeiten benötigen wir zwei verschieden Analysen. Beispiel-Anfragen für diese beiden Analysen sehen wie folgt aus:

\textbf{Suchanfrage:} \textit{Krebs $\rightarrow$ Springermedizin-Suche}

\begin{table}[H]
\centering
\vspace{-.75em}
\caption[Beispiel Analyse-Aufbau Webtrekk-Schnittstelle]{Beispiel Analyse-Aufbau Webtrekk-Schnittstelle}
\vspace{-.5em}
\label{tab:BeispielCTDaten}
\footnotesize
\renewcommand*{\arraystretch}{1.2}
\resizebox{\textwidth}{!}{%
\begin{tabular}{lll}
\hline
\textbf{Parameter}        & \textbf{Wahrscheinlichkeit $P(A_{u})$}                                                                                                                                                            & \textbf{Wahrscheinlichkeit $P(E_{r_{u}})$}                                                                                                                              \\ \hline
\textit{analysis}         & Navigation|Events                                                                                                                                                                                & Navigation|Events                                                                                                                                                      \\ \hline
\textit{time\_start}      & 2016-08-01                                                                                                                                                                                       & 2016-08-01                                                                                                                                                             \\ \hline
\textit{time\_stop}       & 2016-08-31                                                                                                                                                                                       & 2016-08-31                                                                                                                                                             \\ \hline
\textit{column}           & {[}Page Impressions,\% Visits,Ø - Suchtreffer{]}                                                                                                                                                 & {[}Page Impressions,\% Visits,Ø - Suchtreffer{]}                                                                                                                       \\ \hline
\textit{analysis\_filter} & \begin{tabular}[c]{@{}l@{}}{[} {[},Events,LIKE,searchresult-*{]},\\ {[}AND,Internal Search Phrases,LIKE,*Krebs*{]},\\ {[}OR,Internal Search Phrases,LIKE,*Maligne Tumoren*{]} {]}\end{tabular} & \begin{tabular}[c]{@{}l@{}}{[} {[},Events,LIKE,searchresult-1.*{]},\\ {[}OR,Events,LIKE,searchresult-4.*{]},\\ {[}OR,Events,LIKE,searchresult-29.*{]} {]}\end{tabular} \\ \hline
\end{tabular}
}
\vspace{-2em}
\end{table}

In der Antwort für diese Analysen werden die Click-Trough-Daten als JSON-Array\footnote{Die JavaScript Object Notation (JSON) ist ein kompaktes Format zum Austausch von Daten} zurückgegeben. Wie der ausgewertete JSON-Array aussehen kann, haben wir in Kapitel \ref{sec:Grundlagen:Grundbegriffe:Click-Through-Daten:AussehenClick-Through-Daten} angeschaut. 


\section{Zusammenfassung}
\label{sec:Implementierung:Zusammenfassung}

In diesem Kapitel haben wir eine Übersicht über die Architektur und die wichtigsten Highlights der Implementierung gekriegt. Wir wissen nun \textit{wie} der Lösungsansatz des Reranking-Algorithmus implementiert werden kann. Verknüpfen wir dieses Wissen mit dem aus Kapitel \ref{sec:Reranking}, könnten wir den Reranking-Algorithmus nun auch auf eine andere Suche adaptieren. Im nächsten Kapitel wollen wir nun mithilfe der implementierten Suche evaluieren, welche Verbesserung für das Qualitätsmaß der Suche von Springermedizin, der Lösungsansatz bringt. 		% INCLUDE: Kern der Implementierung
% !TEX root = ../thesis-example.tex
%
%************************************************
% Evaluation
%************************************************
\chapter{Evaluation und Auswertung}
\label{sec:Evaluation}

\section{Einführung}
\label{sec:Evaluation:Einfuehrung}

In vorherigen Kapitel haben wir unseren Lösungsansatz in der Springermedizin-Suche implementiert. Wir wissen nun wie und warum wir den Reranking-Algorithmus umgesetzt haben. Was wir bisher nicht wissen ist, \textit{wie gut} er funktioniert. Mithilfe einer Evaluation wollen wir darum nun messen, wie gut die Suchergebnis-Qualität der aktuellen Springermedizin-Suche im Vergleich zur im Zuge dieser Arbeit entwickelten Lösung mit dem Reranking-Algorithmus ist. 

\subsubsection{Ziel der Evaluation}
\label{sec:Evaluation:Einfuehrung:Ziel}

Die Evaluation soll Informationen darüber liefern, wie viel Verbesserung der neue Lösungsansatz bringt. Aus den Ergebnissen wollen wir erkennen, was an dem Lösungsansatz geändert werden muss, damit die Suche wirklich gute Ergebnisse aus Sicht der User liefert und unter welchen Voraussetzungen, sie im Springermedizin-Umfeld eingesetzt werden kann.

\subsubsection{Methodik}
\label{sec:Evaluation:Einfuehrung:Methodik}
Wir werden durch fachliche Experten (Redakteure von Springermedizin) die Suchergebnisse von oft gesuchten Suchanfragen bewerten lassen. Dazu werden wir eine Testumgebung mit einem Evaluations-System und den beiden oben angesprochenen Suchvarianten aufbauen. Die Redakteure sollen bei der Evaluation, zu jeder Suchanfrage die ersten zehn Suchergebnisse analysieren und anhand der Suchsnippets, die Relevanz bewerten. Diese Analyse sollen sie jeweils einmal auf der aktuellen Springermedizin-Suche und einmal auf der Springermedizin-Suche mit dem Reranking-Algorithmus durchführen. Am Ende werden wir die Ergebnisse auswerten und einen Vergleich der beiden Suchvarianten machen. 

\section{Aufbau der Analyse}
\label{sec:Evaluation:Aufbau}

Mithilfe der Relevanz-Bewertungen wollen wir das Qualitätsmaß der Suchvarianten bestimmen. Dazu müssen wir zuerst die \glqq zuverlässigen\grqq{} Bewertungen mittels \textit{Kappa-Koeffizienten} filtern und dann mithilfe der \textit{nDCG-Metrik} auswerten. Der Prozess dazu sieht wie folgt aus:

\begin{figure}[H]
\centering
\vspace{-.5em}
\caption[Prozess der Datenaufbereitung und Metrik der Auswertung]{Prozess der Datenaufbereitung und Metrik der Auswertung}
\vspace{.5em}
\label{fig:SucheSpringerNature}
\includegraphics[width=\linewidth]{gfx/EvaluationDatenaufbereitung}
\vspace{-2em}
\end{figure}

\pagebreak

In diesem Kapitel werden Formeln eingeführt. Folgend eine Legende der wichtigsten Symbole:

\begin{table}[H]
\centering
\vspace{-.5em}
\caption[Legende der wichtigsten Formel-Symbolen für die Evaluation]{Legende der wichtigsten Formel-Symbolen für die Evaluation}
\label{tab:LegendeSymboleFormelnEvaluation}
\vspace{-.5em}
\footnotesize
\renewcommand*{\arraystretch}{1.2}
\resizebox{\textwidth}{!}{%
\begin{tabular}{llll}
\hline
\multicolumn{2}{l}{\textit{\textbf{Kappa}}}                            		& \multicolumn{2}{l}{\textit{\textbf{nDCG}}}                  \\ \hline
\textbf{Bedeutung} & \textbf{Symbol}                                   			& \textbf{Bedeutung} & \textbf{Symbol}                        \\ \hline
$K$				& Cohens-Kappa-Koeffizient                          				& $r$                	& die Anzahl der zu prüfenden Positionen \\
$p_0$			& Anteil tatsächlich beobachteter Übereinstimmungen 	& $i$                	& die zu prüfende Position               \\
$p_e$			& Anteil zufälliger Übereinstimmungen               			& $rel_i$				& Relevanz-Wert der Position $i$             \\
                   & 																						& $REL$				& die Menge der Relevanz-Werte der Dokumente in $r$ in idealer Reihenfolge\\
                   & 																						& $DCG_r$     		& Discounted Cumulative Gain             \\
                   &                                                   								& $IDCG_r$    		& Ideal Discounted Cumulative Gain       \\
                   &                                                   								& $nDCG_r$  		& Normalized Discounted Cumulative Gain  \\ \hline
\end{tabular}
}
\vspace{-2.5em}
\end{table}

\subsection{Datengrundlage}
\label{sec:Evaluation:Aufbau:Datengrundlage}

\subsubsection{Filterung der nutzbaren Daten mittels Cohens Kappa}
\label{sec:Evaluation:Aufbau:Datengrundlage:EvaluationsdatenFiltern}

Um die Zuverlässigkeit der Relevanz-Bewertungen zu messen, werden wir die gleichen Suchterme von jeweils zwei Redakteuren von Springermedizin bewerten lassen. Haben die Relevanz-Bewertungen ein zu geringes Maß an Übereinstimmung, sind sie für die anschliessenden Auswertungen zu wenig zuverlässig und werden darum in dieser nicht verwendet. Das meist verwendete Maß zur Bewertung der Übereinstimmungsgüte ist der \textit{Cohens-Kappa-Koeffizient} $K$~(siehe \cite{Kappa}). Dieser misst den Anteil \textit{übereinstimmender Bewertungen} $p_0$ und berechnet daraus die Zuverlässigkeit der Bewertung. Hierbei müssen wir berücksichtigen, dass die Beurteiler mit einer gewissen Wahrscheinlichkeit auch zufällig zur gleichen Einschätzung gelangen können. Der Cohens-Kappa-Koeffizient korrigiert das Maß an Übereinstimmung um diesen \textit{Zufallsfaktor} $p_e$. Die Berechnungsformel zur Bestimmung des Cohens-Kappa-Koeffizienten sieht wie folgt aus:

\vspace{-1.5em}
\begin{equation}	
	K = \frac{p_0 - p_e}{1 - p_e}
\end{equation}
\vspace{-1.5em}

\paragraph{Die Übereinstimmungen der beiden Redakteure aus einer Übereinstimmungsmatrix lesen}
Um die Stärke der Übereinstimmung der beiden Redakteuren bei der Relevanz-Bewertung eines Suchterms zu messen, erstellen wir zu jedem Suchterm eine Übereinstimmungsmatrix. Diese enthält die vier Relevanzstufen aus der Bewertung. Die Bewertungen der Suchergebnisse ordnen wir diesen Relevanzstufen zu:

\begin{table}[H]
\centering
\vspace{-.5em}
\caption[Übereinstimmungsmatrix von zwei Redakteuren bei der Klassifikation einer Suchanfrage]{Übereinstimmungsmatrix von zwei Redakteuren bei der Klassifikation einer Suchanfrage}
\label{tab:KreuztabelleKappaBerechnung}
\vspace{-.5em}
\resizebox{\textwidth}{!}{%
\setlength{\arrayrulewidth}{.5pt}
\renewcommand*{\arraystretch}{1.2}
\begin{tabular}{llcccccl}
\hhline{|*2{~}|*5{-}|}
                                                                                    & \multicolumn{1}{l|}{\textbf{}}                           & \multicolumn{4}{c|}{\cellcolor[HTML]{C0C0C0}\textbf{Redakteur 2}}                                                                                                                                                                      & \multicolumn{1}{c|}{\cellcolor[HTML]{C0C0C0}\textbf{Gesamt}}                                                                                  &                                 \\
                                                                                    & \multicolumn{1}{l|}{}                                    & \multicolumn{1}{c|}{\cellcolor[HTML]{C0C0C0}\textbf{-1}} & \multicolumn{1}{c|}{\cellcolor[HTML]{C0C0C0}\textbf{0}} & \multicolumn{1}{c|}{\cellcolor[HTML]{C0C0C0}\textbf{1}} & \multicolumn{1}{c|}{\cellcolor[HTML]{C0C0C0}\textbf{2}} & \multicolumn{1}{l|}{\cellcolor[HTML]{C0C0C0}}                                                                                                 &                                 \\ \hhline{|*7{>{\arrayrulecolor{black}}-}|~|}
\multicolumn{1}{|l}{\cellcolor[HTML]{C0C0C0}}                                       & \multicolumn{1}{l|}{\cellcolor[HTML]{C0C0C0}\textbf{-1}} & \multicolumn{1}{c|}{\cellcolor[HTML]{00D2CB}$a$}         & \multicolumn{1}{c|}{$b$}                                & \multicolumn{1}{c|}{$c$}                                & \multicolumn{1}{c|}{$d$}                                & \multicolumn{1}{c|}{$(a+b+c+d)/n$}                                                                                                            & \multicolumn{1}{c}{\textbf{$r_{-1}$}} \\ \hhline{|*1{>{\arrayrulecolor[HTML]{C0C0C0}}-}*6{>{\arrayrulecolor{black}}-}|}
\multicolumn{1}{|l}{\cellcolor[HTML]{C0C0C0}}                                       & \multicolumn{1}{l|}{\cellcolor[HTML]{C0C0C0}\textbf{0}}  & \multicolumn{1}{c|}{$e$}                                 & \multicolumn{1}{c|}{\cellcolor[HTML]{00D2CB}$f$}        & \multicolumn{1}{c|}{$g$}                                & \multicolumn{1}{c|}{$h$}                                & \multicolumn{1}{c|}{\cellcolor[HTML]{FFFFFF}$(e+f+g+h)/n$}                                                                                    & \multicolumn{1}{c}{\textbf{$r_0$}} \\ \hhline{|*1{>{\arrayrulecolor[HTML]{C0C0C0}}-}*6{>{\arrayrulecolor{black}}-}|}
\multicolumn{1}{|l}{\cellcolor[HTML]{C0C0C0}}                                       & \multicolumn{1}{l|}{\cellcolor[HTML]{C0C0C0}\textbf{1}}  & \multicolumn{1}{c|}{$i$}                                 & \multicolumn{1}{c|}{$j$}                                & \multicolumn{1}{c|}{\cellcolor[HTML]{00D2CB}$k$}        & \multicolumn{1}{c|}{$l$}                                & \multicolumn{1}{c|}{$(i+j+k+l)/n$}                                                                                                            & \multicolumn{1}{c}{\textbf{$r_1$}} \\ \hhline{|*1{>{\arrayrulecolor[HTML]{C0C0C0}}-}*6{>{\arrayrulecolor{black}}-}|}
\multicolumn{1}{|l}{\multirow{-4}{*}{\cellcolor[HTML]{C0C0C0}\textbf{Redakteur 1}}} & \multicolumn{1}{l|}{\cellcolor[HTML]{C0C0C0}\textbf{2}}  & \multicolumn{1}{c|}{$m$}                                 & \multicolumn{1}{c|}{$n$}                                & \multicolumn{1}{c|}{$o$}                                & \multicolumn{1}{c|}{\cellcolor[HTML]{00D2CB}$p$}        & \multicolumn{1}{c|}{$(m+n+o+p)/n$}                                                                                                            & \multicolumn{1}{c}{\textbf{$r_2$}} \\ \hhline{|*7{-}|}
\multicolumn{2}{|l|}{\cellcolor[HTML]{C0C0C0}\textbf{Gesamt}}                                                                                  & \multicolumn{1}{c|}{$(a+e+i+m)/n$}                       & \multicolumn{1}{c|}{$(b+f+j+n)/n$}                      & \multicolumn{1}{c|}{$(c+g+k+o)/n$}                      & \multicolumn{1}{c|}{$(d+h+l+p)/n$}                      & \multicolumn{1}{c|}{\cellcolor[HTML]{FFCB2F}\begin{tabular}[c]{@{}c@{}}$\textbf{n} = \sum \text{ aller Matrixelemente} \left( a,b...o,p \right)$ \end{tabular}} &                                 \\ \hhline{|*7{-}|}
                                                                                    &                                                          & \textbf{$c_{-1}$}                                              & \textbf{$c_0$}                                             & \textbf{$c_1$}                                             & \textbf{$c_2$}                                             & \multicolumn{1}{l}{}                                                                                                                          &                                
\end{tabular}%
}
\vspace{-2em}
\end{table}

Den Anteil tatsächlich beobachteter Übereinstimmungen $p_0$ können wir direkt aus den Werten der \textit{Hauptdiagonalen} der Matrix berechnen. Die Berechnungsformel dazu sieht wie folgt aus: 

\vspace{-1.5em}
\begin{equation}	
	p_0 = \frac{ \sum \text{ der Übereinstimmungen} \left( a + f + k + p \right)}{ \sum \text{ aller Übereinstimmungen} \left( n \right) }
\end{equation}
\vspace{-1em}

Den daraus resultierenden Wert, müssen wir um den Anteil zufälliger Übereinstimmungen $pe$ korrigieren. Der Wert von $pe$ wird mithilfe der Randsummen der Matrix (Spalten- bzw. Zeilensummen) berechnet. Dazu muss jede Randsumme zuerst durch $n$ dividiert werden. Danach wird zu jeder Kategorie das \textit{Produkt der Spalten und Zeilensumme} gebildet, mit welchem anschließend die Summe der Kategorien berechnet wird. Die komplette Berechnungsformel von $pe$ sieht wie folgt aus:

\vspace{-1.5em}
\begin{equation}	
	p_e = \left(  \left(\frac{  r_{-1} }{ n } \cdot \frac{ c_{-1} }{ n } \right) +   \left(\frac{  r_0 }{ n } \cdot \frac{ c_0  }{ n } \right) +   \left(\frac{  r_1 }{ n } \cdot \frac{ c_1 }{ n } \right)  +  \left(\frac{ r_2 }{ n } \cdot \frac{   c_2 }{ n }\right) \right)
\end{equation}
\vspace{-2em}

\paragraph{Kappa-Koeffizienten gewichten um Abweichungen der Bewertungen in Relation zu derer Relevanzdifferenz zu stellen}
Relevanz-Bewertungen werden meist sehr subjektiv gefällt. Wir müssen darum davon ausgehen, dass die Redakteure häufig kleinere Abweichungen in den Bewertungen haben werden. Weichen sie mehrere Kategorien voneinander ab, sollten wir diese Abweichung schwerer wiegen als die, bei benachbarten Kategorien. Unsere bisherige Formel stuft alle Abweichungen gleich ein. Mithilfe der Erweiterung des Kappa-Koeffizienten um eine Gewichtung der Abweichungsstärke zwischen 0 und 1, können wir unsere Formel in das gewichtete Kappa $K_w$~(siehe \cite{KappaWerte}) transformieren. Dazu müssen wir bei der Berechnung der Spalten- und Zeilensummen \textit{die Anzahl der Kategorien}, die das Matrixelement zur Hauptdiagonalen abweicht, berücksichtigen und diese Abweichung gewichten. Für unsere Berechnung definieren wir die Gewichte wie folgt:

\begin{table}[H]
\centering
\vspace{-.5em}
\caption[Gewichtung der Abweichungsstärke einer Kategorie zur Hauptdiagonalen]{Gewichtung der Abweichungsstärke einer Kategorie zur Hauptdiagonalen}
\label{tab:GewichtungAbweichungenKappaBerechnung}
\vspace{-.5em}
\footnotesize
\renewcommand*{\arraystretch}{1.2}
\begin{tabular}{cl}
\hline
\multicolumn{1}{l}{\textbf{Gewichtung}} & \textbf{Anzahl Kategorien Abweichung} \\ \hline
\textit{0.25}                                     & 0 \textit{(auf Hauptdiagonal)}                 \\ 
\textit{0.5}                                      & 1 \textit{(benachbarte Kategorie)}             \\ 
\textit{0.75}                                     & 2                                     \\ 
\textit{1}                                        & 3                                     \\ \hline
\end{tabular}
\vspace{-1.5em}
\end{table}

\paragraph{Kappa interpretieren}

Wie in \cite{KappaWerte} beschrieben, müssen die Kappa-Werte individuell interpretiert werden. Es gibt jedoch Richtlinien. In \cite{Kappa} wird bei einem Kappa-Wert ab 0.60 von einer guten Übereinstimmung ausgegangen. Da wir bei unserer Evaluation nur mit einer begrenzten Anzahl von Bewertungen arbeiten können, müssen wir das Mindestmaß der Übereinstimmungsgüte anhand der Kappa-Koeffizienten der Bewertungen definieren. Wir werden es so definieren, dass wir mindestens 80 Prozent der Bewertungen auswerten können. Die unter dem Mindestmaß liegenden Bewertung, werden wir in der Auswertung ignorieren. 

\subsection{Metrik}
\label{sec:Evaluation:Aufbau:Metrik}

\subsubsection{Qualitätsmaß einer Suchvariante bestimmen}
\label{sec:Evaluation:Aufbau:Metrik:QualitaetMessen}

In der Evaluation werden zu jedem Suchterm zwei Bewertungen für die aktuelle Springermedizin-Suche und zwei Bewertungen für die Springermedizin-Suche mit dem Reranking-Algorithmus abgegeben. Um das Qualitätsmaß einer Suche zu bestimmen, werden wir den \textit{Normalized Discounted Cumulative Gain} (nDCG, siehe \cite{nDCG}) verwenden. Der nDCG misst die Qualität des Rankings der Suche und wird in der Information Retrieval\footnote{Mit Information Retrieval werden Methoden und Verfahren, die der Aufbereitung und Speicherung von Wissen und der Gewinnung von Informationen dienen bezeichnet} oft eingesetzt um die Effektivität eines Such-Algorithmus zu messen.  Um das Qualitätsmaß der beiden in der Evaluation verwendeten Suchvarianten zum Suchterm zu bestimmen, berechnen wir zu jeder Suchvariante den nDCG der beiden Bewertungen des Suchterms. Nehmen wir den Mittelwert der beiden resultierenden nDCG-Werte, erhalten wir den effektiven nDCG-Wert. Die nDCG-Werte der beiden Suchvarianten können wir dann miteinander vergleichen.

\subsubsection{Evaluationsdaten mittels nDCG auswerten}
\label{sec:Evaluation:Aufbau:Metrik:EvaluationsdatennDCG}

Der nDCG verfolgt die Grundidee, Suchergebnislisten dahingehend zu untersuchen, ob Dokumente mit hoher Relevanz zum Suchterm, vor denen mit weniger Relevanz stehen. Der nDCG vergleicht dazu die Relevanz der Dokumente mit ihrer Reihenfolge im Suchresultat. Diese Metrik macht für unseren Reranking-Algorithmus insofern Sinn, dass sie nicht von bestimmten Relevanz-Werten für die Suchresultate ausgeht, sondern wie unser Algorithmus, sich nur auf die Reihenfolge der Relevanz-Werte konzentriert. 

\subsubsection{Berechnung des Qualitätsmaß einer Suchvariante mittels nDCG}
\label{sec:Evaluation:Aufbau:Metrik:BerechnungnDCG}

\paragraph{Der nDCG baut auf dem DCG auf} Um den nDCG-Wert zu bestimmen, müssen wir zuerst den \textit{Discount Cumulative Gain} (DCG) jeder Position des untersuchten Suchresultats berechnen. Der DCG zielt darauf ab, das Qualitätsmaß einer Suchergebnisliste herunterzustufen, wenn relevante Dokumente schlechter, als weniger relevante positioniert sind. Die Formel des DCG wird wie folgt definiert:

\vspace{-1.5em}
\begin{equation}	
DCG_{r} = \sum\limits_{i=1}^r \frac{2^{rel_{i}} - 1}{\log_2(i+1)}
\end{equation}
\vspace{-1em}

Wie wir sehen, werden Dokumente umso geringer bewertet, je weiter hinten sie im Suchresultat erscheinen. Dafür sorgt $\log_2(i+1)$ als Divisor in der Summenfunktion, dessen Wert mit steigendem Wert der Dokumentposition größer wird.

\paragraph{Der DCG muss normalisiert werden, um das Qualitätsmaß einer Suche über unterschiedliche Anfragen zu bewerten}
Der DCG ist auf den Vergleich von Suchanfragen mit gleicher Resultatslänge ausgelegt. Haben die zu untersuchenden Suchanfragen, eine unterschiedliche Anzahl der zu untersuchenden Positionen, variiert die maximal erreichbare Punktzahl. Der nDCG normalisiert diese Punktewerte, indem er die Reihenfolge der Relevanz-Werte des Suchergebnisses mit der \textit{idealen Reihenfolge} derselben Relevanz-Werte vergleicht. Die Formel des nDCG wird wie folgt definiert:

\vspace{-.25em}
\begin{spacing}{.25}
\begin{align}
  &&	nDCG_{r} &= \frac{DCG_r}{IDCG_r} &\\
  \intertext{wobei:}
  &&	IDCG_{r} 	&= \sum\limits_{i=1}^{\vert REL \vert} \frac{2^{rel_{i}} - 1}{\log_2(i+1)}
\end{align}
\end{spacing}
\vspace{.25em}

Der resultierende Ergebnis-Wert des $nDCG_r$ bewegt sich zwischen 0 und 1. Entspricht die Reihenfolge der Suchergebnisse, der Relevanz der Suchergebnisse, so gilt $DCG_r = IDCG_r$. Dies entspricht dem Idealfall, die Suche besitzt in diesem Fall den $nDCG_r$-Wert 1 und somit das maximal mögliche Qualitätsmaß für den getesteten Suchterm.

\subsection{Vorgehen}
\label{sec:Evaluation:Aufbau:Vorgehen}

\subsubsection{Testumgebung aufbauen}
\label{sec:Evaluation:Aufbau:Vorgehen:Aufbau}

Um eine Evaluation durchführen zu können, müssen wir eine passende Testumgebung aufbauen. Diese besteht aus einem Evaluations-System, einer Instanz der aktuellen Springermedizin-Applikation und einer Instanz des neu implementierten Lösungsansatzes. Auf dem Evaluations-System sollen fachliche Experten (Redakteure von Springermedizin) die Relevanz der Suchergebnisse der beiden Suchmaschinen vergleichen. Dazu sollen die jeweils ersten zehn Suchergebnisse nach Relevanz zum Suchterm bewertet werden. Der Ergebnisse werden in einer Datenbank gespeichert, um sie später auszuwerten. 

\paragraph{Aufbau des Evaluations-Systems}
Für die Analysen der Beurteiler implementieren wir selber ein kleines Evaluations-System als Web-Applikation. Das hat den Vorteil, dass wir selber definieren können, wie der Analyse-Prozess und wie die Datenstruktur der Analyse-Ergebnisse aussehen sollen. Das System umfasst eine Administrationsoberfläche zur Verwaltung der Analysedaten und eine Anwenderoberfläche zur Analyse der zugeteilten Suchterme. Der Beurteiler soll nur seine, ihm zugeteilten Suchterm-Analysen sehen. Um Analysen durchführen zu können, muss sich der Beurteiler darum an der Applikation anmelden. So können wir jedem Beurteiler ein eigenes Profil anlegen und ihm seine zu analysierenden Suchterme zuweisen. 

\paragraph{Analyse eines Suchterms}
Wir wollen zu jedem Suchterm eine Bewertung der aktuellen Springermedizin-Suche und eine, der Suche mit dem implementierten Reranking-Algorithmus. Wir bauen die Analysen darum so, dass der Benutzer beide Bewertungen nacheinander, in derselben Analyse ausführt. Zur Bewertung implementieren wir ein Maske, bestehend aus der zu beurteilende Suchvariante und einem Formular zur Bewertung der ersten zehn Suchresultate des Suchergebnisses. Die Bewertung besteht aus einer Skala von vier Relevanz-Werten, wobei \textit{nicht relevante} Ergebnisse bestraft werden sollen und darum einen negativen Relevanz-Wert erhalten:

\begin{table}[H]
\centering
\vspace{-.5em}
\caption[Relevanz-Werte für Bewertung der Suchresultate]{Relevanz-Werte für Bewertung der Suchresultate}
\label{tab:RelevanzWerteBewertungEvaluation}
\vspace{-.5em}
\footnotesize
\renewcommand*{\arraystretch}{1.2}
\begin{tabular}{llc}
\hline
\textbf{Relevanz}            & \textbf{Beschreibung}           & \textbf{Relevanz-Wert} \\ \hline
\textit{not relevant}        & Ergebnis hat \textbf{\textit{gar keine}} Relevanz & -1 \\
\textit{moderately relevant} & Ergebnis ist \textbf{\textit{eher irrelevant}}    & 0 \\
\textit{relevant}            & Ergebnis ist \textbf{\textit{eher relevant}}      & 1 \\
\textit{highly relevant}     & Ergebnis ist \textbf{\textit{sehr relevant}}      & 2 \\ \hline
\end{tabular}
\vspace{-2em}
\end{table}

Ist eine Bewertung gespeichert, wird die zweite Variante der Suche geladen. Eine Analyse ist dann abgeschlossen, wenn beide Suchvarianten bewertet sind. Der Ablauf einer Analyse sieht wie folgt aus:

\begin{figure}[H]
\centering
\vspace{-.5em}
\caption[Analyseprozess für Bewertung einer Suchvariante]{Analyseprozess für Bewertung einer Suchvariante}
\vspace{.5em}
\label{fig:AnalysemaskeEvaluation}
\includegraphics[width=\linewidth]{gfx/AnalysemaskeEvaluation}
\vspace{-1em}
\end{figure}

\subsubsection{Evaluations-Daten speichern}
\label{sec:Evaluation:Aufbau:Vorgehen:Speichern}

Zu jeder Analyse werden wir neben den Relevanz-Bewertungen und den Suchterm-Informationen auch wichtige Informationen zu den Click-Through-Daten und den Suchergebnissen speichern. Folgende Informationen liegen uns am Ende einer Analyse vor:

\begin{table}[H]
\centering
\vspace{-.5em}
\caption[Gespeicherte Evaluations-Daten zur Suchterm-Analyse]{Gespeicherte Evaluations-Daten zur Suchterm-Analyse}
\label{tab:EvaluationsDatenSpeicherung}
\vspace{-.5em}
\footnotesize
\renewcommand*{\arraystretch}{1.2}
\resizebox{\textwidth}{!}{%
\begin{tabular}{llll}
\hline
\textbf{Objekt}                                        & \textbf{Beide Suchvarianten}             & \textbf{Suche mit Reranking-Algorithmus} & \textbf{Beschreibung}                              \\ \hline
\multirow{4}{*}{\textit{\textbf{Analyse}}}             & \textit{Suchterm}                        & \textit{}                                & Analysierter Suchterm                              \\
                                                       & \textit{Beurteiler}                      & \textit{}                                & User-Informationen des Beurteilers                 \\
                                                       & \textit{Suchvariante}                    & \textit{}                                & Suche mit / ohne Reranking-Algorithmus             \\
                                                       & \textit{}                                & \textit{Wert des Zufallsfaktors}         & Einfluss-Wert Zufallsfaktor                        \\ \hline
\textit{\textbf{Bewertungen}}                          & \textit{Relevanz-Werte der Positionen}   & \textit{}                                & Relevanz-Bewertungen der ersten zehn Suchresultate \\ \hline
\multirow{2}{*}{\textit{\textbf{Click-Through-Daten}}} & \textit{}                                & \textit{User-Filter}                     & Alle Benutzer / eingeloggte Benutzer der Suche     \\
                                                       & \textit{}                                & \textit{Anzahl Klicks Gesamt}            & Anzahl gelesener Klicks aus Click-Through-Daten    \\ \hline
\textit{\textbf{Suchergebnis}}                         & \textit{Dokument-ID's der Suchresultate} & \textit{}                                & Dokument-ID's der ersten zehn Suchergebnisse       \\ \hline
\end{tabular}
}
\vspace{-2em}
\end{table}

\subsubsection{Evaluations-System auswerten}
\label{sec:Evaluation:Aufbau:Vorgehen:Auswerten}

Nach Abschluss der Evaluationsphase werden wir die Evaluations-Daten auswerten. Die Auswertung der Daten findet direkt im Evaluations-System statt. Dazu werden wir die Relevanz-Bewertungen aus der Datenbank lesen und wie oben beschrieben mit dem Cohens-Kappa-Koeffizienten und der nDCG-Metrik auswerten. Mithilfe der Evaluations-Daten können wir dann auch weitere Auswertungen zu den Click-Through-Daten und den Suchergebnissen der Suchterme machen.

\subsection{Durchführung}
\label{sec:Evaluation:Aufbau:Durchfuehrung}

\subsubsection{Zu analysierende Suchterme}
\label{sec:Evaluation:Aufbau:Durchfuehrung:Aufgabenstellung}

Um die Evaluation praxisrelevant und mit genügend Click-Through-Daten ausführen zu können, wurden die Suchanfragen der letzten zwei Monate auf der Springermedizin-Suche analysiert und 80 oft gesuchte Suchterme ausgewählt. Springermedizin hat ein breites Spektrum an medizinischen Fachgebieten. Um fachlich gute Bewertungen zu kriegen, haben wir die Suchterme nach Fachrichtung unterteilt und jeweils zwei Beurteilern zugeteilt, die aus der Fachrichtung kommen:

\begin{table}[H]
\centering
\vspace{-.5em}
\caption[Fachrichtungen der Suchterme der Evaluation]{Fachrichtungen der Suchterme der Evaluation}
\label{tab:FachrichtungenSuchtermeEvaluation}
\vspace{-.5em}
\footnotesize
\renewcommand*{\arraystretch}{1.2}
\resizebox{\textwidth}{!}{%
\begin{tabular}{llc}
\hline
\textbf{Fachrichtung}            		& \textbf{Medizinisches Gebiet / Medizinischer Zweig}                  	& \multicolumn{1}{l}{\textbf{Anzahl zugeteilte Suchterme}} \\ \hline
\textit{Onkologie}               		& Tumorerkrankungen                                                    					& 10                                                       \\ \hline
\textit{Zahnmedizin}             		& Fachgebiet der Zahn-, Mund- und Kieferheilkunde                      		& 20                                                       \\ \hline
\textit{Gynäkologie}             		& Spezifischen Erkrankungen des weiblichen Körpers                    		& 10                                                       \\ \hline
\textit{AINS}                    			& Anästhesiologie, Intensivmedizin, Notfallmedizin und Schmerztherapie & 10                                                       \\ \hline
\textit{Neurologie, Psychiatrie} 	& Erkrankungen des Nervensystems und der Psyche                        	& 10                                                       \\ \hline
\textit{Innere Medizin}          		& Erkrankungen der inneren Organe                                      			& 10                                                        \\ \hline
\textit{Orthopädie}              		& Aufbau der Knochen und Muskeln des Menschen                          	& \multirow{3}{*}{10}                                      \\
\textit{Urologie}               			& Erkrankungen der Niere, Harnblase, Harnleiter und Harnröhre         	&                                                          \\
\textit{HNO}               				& Hals-Nasen-Ohren-Heilkunde                                           				&                                                          \\ \hline
\textbf{Gesamt:}                 		& \textbf{}                                                            							& \textbf{80}                                              \\ \hline
\end{tabular}
}
\vspace{-2em}
\end{table}

\subsubsection{Aufgabenstellung der Analyse}
\label{sec:Evaluation:Aufbau:Durchfuehrung:Aufgabenstellung}

Die Aufgabe der Analyse besteht darin, die jeweils ersten zehn Suchergebnisse nach Relevanz zum Suchterm zu bewerten. Insgesamt, soll jeder Beurteiler die ihm zugeteilten 20 Suchterme analysieren. Eine Analyse beinhaltet jeweils eine Bewertung des Suchterms, auf beiden Suchvarianten. In welcher Reihenfolge die Suchvarianten den Beurteilern während der Analyse präsentiert werden, ist rein \textit{zufällig} und \textit{variiert}. Es soll nicht ersichtlich sein, welche Variante jeweils bewertet wird. Dadurch können wir ausschließen, dass der Beurteiler durch die Bekanntgabe der Suchvariante subjektiv beeinflusst wird. 

\subsubsection{Verschiedene Varianten des neuen Lösungsansatzes evaluieren}
\label{sec:Evaluation:Aufbau:Durchfuehrung:EvaluationsdatenVarianteLoesungsansatzes}

Wie wir wissen, hat der Reranking-Algorithmus einige Faktoren, die variabel definiert werden können. Zu diesen Faktoren gehört der Einfluss des Zufallswertes in das Reranking ($\lambda$) und der \glqq Login-Status der User\grqq{} während der Suche, für die Selektion der Click-Through-Daten. Wir werden darum verschiedene Konstellationen testen, um die optimale Konstellation finden zu können. 

\paragraph{Definition der variablen Faktoren für den Reranking-Algorithmus}
Für den \textit{Einfluss des Zufallsfaktors} werden wir zwei Werte für $\lambda$ definieren. Diese sind 0.1 (zehn Prozent) und 0.01 (ein Prozent). Für die \textit{Selektion der Click-Through-Daten} werden wir zwischen an der Applikation angemeldeten Benutzern und allen Benutzern (\textit{inkl. anonymen Benutzern}) unterscheiden. Aus den beiden Einflusswerten des Zufallsfaktors und der Unterscheidung zwischen angemeldeten und allen Benutzern, ergeben sich vier Konstellationen. Jeder Konstellation werden wir jeweils 25 Prozent der Suchterme zuteilen. Die Zuteilung der Suchterme werden wir mithilfe des Evaluations-Systems zufällig generieren lassen:

\begin{figure}[H]
\centering
\vspace{-.5em}
\caption[Aufteilung der Analysen für Evaluation]{Aufteilung der Analysen für Evaluation}
\vspace{.5em}
\label{fig:AufteilungAnalysenEvaluation}
\includegraphics[width=0.9\linewidth]{gfx/EvaluationsvariantenRerankingSuche}
\vspace{-2em}
\end{figure}

\section{Auswertung der Suchergebnis-Qualität}
\label{sec:Evaluation:Auswertung}

\subsection{Quantitative Auswertung}
\label{sec:Evaluation:Auswertung:QuantitativeAuswertung}

\subsection{Diskussion}
\label{sec:Evaluation:Auswertung:Diskussion}

\section{Zusammenfassung}
\label{sec:Evaluation:Zusammenfassung}

%--------------------

%\paragraph{Manche Suchbegriffe sind aus meiner Sicht zu allgemein ( „operative Therapie“) oder veraltet („Prostata-Adenom“ statt benigne „Prostatahyperplasie“)%
			% INCLUDE: Evaluation
% !TEX root = ../thesis-example.tex
%
%************************************************
% Zusammenfassung und Ausblick
%************************************************
\chapter{Zusammenfassung und Ausblick}
\label{sec:ZusammenfassungAusblick}

Im letzten Kapitel haben 

\section{Zusammenfassung}
\label{sec:ZusammenfassungAusblick:Zusammenfassung}

\paragraph{Ausgangspunkt}

\paragraph{Das Ziel der Arbeit}

\paragraph{Das Ergebnis}


\paragraph{Die Erkenntnisse}

\section{Ausblick}
\label{sec:ZusammenfassungAusblick:Ausblick}






Grundsätzlich konnten wir feststellen dass die Verwertung der Click-Through-Daten tendenziell ein guter Weg zur Optimierung der Suche sein kann, vorausgesetzt deren Relevanz-Feedback ist aussagekräftig.  		% INCLUDE: Zusammenfassung

\cleardoublepage

%************************************************
% Anhang
%************************************************
\addcontentsline{toc}{chapter}{Anhang}
\addcontentsline{toc}{section}{Literatur-Verzeichnis}
{%
\setstretch{1.1}
\renewcommand{\bibfont}{\normalfont\small}
\setlength{\biblabelsep}{0pt}
\setlength{\bibitemsep}{0.5\baselineskip plus 0.5\baselineskip}
\printbibliography[nottype=online]
\printbibliography[heading=subbibliography,title={Webseiten},type=online,prefixnumbers={@}]
}
\cleardoublepage

\addcontentsline{toc}{section}{Abbildungs-Verzeichnis}
\listoffigures
\cleardoublepage

\addcontentsline{toc}{section}{Tabellen-Verzeichnis}
\listoftables
\cleardoublepage

%\clearpage
%\newpage

% **************************************************
% End of Document CONTENT
% **************************************************
\end{document}
