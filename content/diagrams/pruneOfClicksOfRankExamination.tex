
In der unten folgenden Analyse sehen wir das positionsbezogene Klick-Verhalten der User auf der Springermedizin-Suche. Dazu wurden über den Zeitraum von einem Monat, die letzten 1000 Suchanfragen ausgewertet. Dargestellt sehen wir die Häufigkeitsverteilung der Klicks als Graph. Wir beschränken uns hierbei auf die ersten 20 Positionen der Suchresultate. Wie wir sehen, nimmt die Anzahl der Klicks mit zunehmender Position exponentiell ab. Dieser Effekt kann ebenfalls, wie in Abb. \ref{fig:Grundlagen:AnalyseKlicksTop10Suchergebnisse}, durch das Potenzgesetz (Power Law, siehe \cite{PowerLaw}) beschrieben werden. 

\begin{figure}[H]
\centering 
\vspace{-1em}
\caption[Analyse der Klicks auf die ersten 20 Positionen der Suchergebnisse aller Suchanfragen. \textit{Zeitraum der Analyse: 19.08.16 - 19.09.16}]{Analyse der Klicks auf die ersten 20 Positionen der Suchergebnisse aller Suchanfragen. \\ \textit{Zeitraum der Analyse: 19.08.16 - 19.09.16}}
\label{fig:Grundlage:AnalyseKlicksPositionen}

\pgfplotstableread[col sep=semicolon]{content/diagrams/clicks_top1000_ranks_result.csv}\topRanks
  
\begin{tikzpicture}
\begin{axis}[
	width=14cm,
	height=3cm,
	scale only axis,
	xmajorgrids,
	xminorgrids,
    ylabel=\textbf{Anzahl der Klicks}, 
	xlabel=\textbf{Position im Suchergebnis},
	nodes near coords, 
	 every node near coord/.append style={xshift=+10pt,yshift=-1pt},
    xtick=data,
    ymin=0,
    xmin=1,
    xmax=20,
    legend style={font=\tiny}
]
\addplot table [
    x=Position,
    y=Klicks
] {\topRanks};
\legend{Anzahl Klicks}
\end{axis}
\end{tikzpicture}

\vspace{-2em}
\end{figure}

Betrachten wir den Graphen, sehen wir, dass besonders die erste Position, auffällig oft angeklickt wird. Daraus könnten wir die Vermutungen ableiten, dass die Suche eine sehr gute Qualität besitzt, weil die zu oberst angezeigten Dokumente, sehr relevant sind und die meisten User der Suchmaschine vertrauen. Wie wir aus den Analysen von \cite{Joachims} lesen können, müssen wir davon ausgehen, dass die Häufigkeit des Klicks auf die ersten Positionen des Suchresultates eher dem Vertrauen der User der Suchmaschine, als der Qualität der Suche geschuldet ist. Vergleichen wir die Analyse aus Abb. \ref{fig:Grundlagen:AnalyseKlicksTop10Suchergebnisse} mit dieser Analyse, sehen wir ein sehr ähnliches Muster in der Häufigkeitsverteilung der Klicks. Wir können anhand der Klick-Zahlen ebenfalls vermuten, dass die am häufigsten angeklickten Dokumente, sich dabei auf den ersten Positionen des Suchergebnisses befunden haben.