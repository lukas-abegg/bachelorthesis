\begin{minipage}{0.40\linewidth}
Auf der rechts dargestellten Abb. beschreibt die x-Achse die Anzahl der gefundenen Klicks, für die Berechnung des CTR mittels Reranking-Algorithmus. Die y-Achse beschreibt den nDCG-Wert, der aus dem Reranking-Algorithmus resultierenden Suchergebnisliste. Das Verhalten der nDCG-Werte abhängig der Anzahl Klicks, zeigt ein sehr sprunghaftes Verhalten, wobei sich die nDCG-Werte in den meisten Fällen im Bereich zwischen 0.7 und 1 bewegen. Ein sehr schwacher nDCG-Wert ist die Ausnahme und nur bei der Klickanzahl 65 zu sehen.
\end{minipage}
\hfill
\begin{minipage}{0.55\linewidth}
\begin{figure}[H]
\centering 
\vspace{-1.5em}
\caption[Click-Through-Daten: Einfluss der Anzahl Klicks gesamt auf $nDCG$]{Click-Through-Daten: Einfluss der Anzahl Klicks gesamt auf $nDCG$}
\label{fig:Evaluation:Auswertung:ClicksNDCG}

\footnotesize
\pgfplotstableread[col sep=semicolon]{content/diagrams/eval/clicksNDCG.csv}\clicks
\pgfsetplotmarksize{.5pt}
  
\begin{tikzpicture}
\begin{axis}[
	width=8cm,
	height=3.5cm,
	scale only axis,
	xmajorgrids,
	xminorgrids,
    ylabel=\textbf{$nDCG$}, 
	xlabel=\textbf{Anzahl Klicks},
    xtick={0,20,40,60,80,100},
   	xmin=0,
	xmax=100,
    xlabel shift = 1in,
    ylabel shift = 1in,
    y label style={yshift=-4mm},
    legend style={font=\tiny},
    legend style={at={(axis cs:100,0)},anchor=south east}
]
\addplot table [
    y=AVG,
    x=Klicks
] {\clicks};
\addplot table [
    y=NDCG,
    x=Klicks
] {\clicks};
\legend{Durchschnittlicher $nDCG$, $nDCG$ abhängig der Anzahl Klicks gesamt}
\end{axis}
\end{tikzpicture}

\vspace{-2.5em}
\end{figure}
\end{minipage}


\begin{minipage}{0.40\linewidth}
Bei der rechts folgenden Abb. werden wir wie in der oberen Abb. \ref{fig:Evaluation:Auswertung:ClicksNDCG}, die Einflüsse der Click-Through-Daten messen. Nur dieses Mal vergleichen wir die Diversität der angeklickten Positionen, mit deren Einfluss auf den nDCG. Wie wir an der roten Linie erkennen, haben die Click-Through-Daten mit weniger als 20 unterschiedlichen Positionen, tendenziell die besten nDCG-Werte. Die Positionszahlen dahinter verhalten sich sehr sprunghaft und weisen viele schwache nDCG-Werte auf.
\end{minipage}
\hfill
\begin{minipage}{0.55\linewidth}
\begin{figure}[H]
\centering 
\vspace{-1em}
\caption[Click-Through-Daten: Einfluss der Anzahl unterschiedlicher Positionen auf nDCG]{Click-Through-Daten: Einfluss der Anzahl unterschiedlicher Positionen auf nDCG}
\label{fig:Evaluation:Auswertung:RanksNDCG}

\footnotesize
\pgfplotstableread[col sep=semicolon]{content/diagrams/eval/ranksNDCG.csv}\ranks
\pgfsetplotmarksize{.5pt}
  
\begin{tikzpicture}
\begin{axis}[
	width=8cm,
	height=3.5cm,
	scale only axis,
	xmajorgrids,
	xminorgrids,
    ylabel=\textbf{$nDCG$}, 
	xlabel=\textbf{Anzahl unterschiedlicher Positionen},
    xtick={0,20,40,60,80},
   	xmin=0,
	xmax=80,
    xlabel shift = 1in,
    ylabel shift = 1in,
    y label style={yshift=-4mm},
    legend style={font=\tiny},
    legend style={at={(axis cs:80,0)},anchor=south east}
]
\addplot table [
    y=AVG,
    x=Positionen
] {\ranks};
\addplot table [
    y=NDCG,
    x=Positionen
] {\ranks};
\legend{Durchschnittlicher $nDCG$, $nDCG$ abhängig der Anzahl unterschiedlicher Positionen}
\end{axis}
\end{tikzpicture}

\vspace{-2em}
\end{figure}
\end{minipage}
