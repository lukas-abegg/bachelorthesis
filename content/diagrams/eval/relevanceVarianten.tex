
\begin{figure}[H]
\centering 
\vspace{-1em}
\caption[Durchschnittliche Relevanz-Werte der Suchvarianten für die ersten zehn Suchpositionen]{Durchschnittliche Relevanz-Werte der Suchvarianten für die ersten zehn Suchpositionen}
\label{fig:Evaluation:Auswertung:RelevanzVarianten}

\footnotesize
\pgfplotstableread[col sep=semicolon]{content/diagrams/eval/relevanceVarianten.csv}\rel
  
\begin{tikzpicture}
\begin{axis}[
	width=11cm,
	height=4cm,
	scale only axis,
	xmajorgrids,
	xminorgrids,
    ylabel=\textbf{Relevanz-Wert}, 
	xlabel=\textbf{Position},
    xtick=data,
    xlabel shift = 1in,
    ylabel shift = 1in,
    xmin=1,
	xmax=10,
    y label style={yshift=-4mm},
   	legend pos=outer north east,
    legend style={font=\tiny},
    legend style={cells={align=left}}
]
\addplot table [
    y=N,
    x=Position
] {\rel};
\addplot table [
    y=R1,
    x=Position
] {\rel};
\addplot table [
    y=R2,
    x=Position
] {\rel};
\addplot table [
    y=R3,
    x=Position
] {\rel};
\addplot table [
    y=R4,
    x=Position
] {\rel};
\legend{aktuelle Springermedizin-Suche, {Reranking-Algorithmus mit:\\$X_u: 0.05$ \\ CTR aller \textit{angemeldeten User}}, {Reranking-Algorithmus mit:\\$X_u: 0.05$ \\ CTR \textit{aller User}}, {Reranking-Algorithmus mit:\\$X_u: 0.1$ \\ CTR aller \textit{angemeldeten User}}, {Reranking-Algorithmus mit:\\$X_u: 0.1$ \\ CTR \textit{aller User}}}
\end{axis}
\end{tikzpicture}

\vspace{-2em}
\end{figure}

Die oben sichtbare Darstellung eines Liniendiagramms, stellt die ersten zehn Positionswerte aller Suchvarianten miteinander ins Verhältnis. Die x-Achse beschreibt hierbei die dargestellte Position im Suchergebnis. Die y-Achse zeigt den Relevanz-Wert, mit welchem die Suchergebnisse auf dieser Position durchschnittlich bewertet werden. Der Relevanz-Wert kann einen Wert von minimal 0 bis und maximal 3 annehmen. Die Relevanz-Werte der Suchvarianten bewegen sich in einem Bereich zwischen 1 und 2.5. Die farbigen Linien beschreiben jeweils die Relevanz-Bewertungen für entweder die Springermedizin-Suche oder eine der verschiedenen Suchvarianten der Suche mit dem Reranking-Algorithmus. 

\paragraph{Suchvarianten mit CTR der angemeldeten User weist die besten Relevanz-Werte auf}
Wie bereits in der Auswertung des nDCG in Abb. \ref{fig:Evaluation:Auswertung:NDCGVarianten}, weist die Suchvariante des Reranking-Algorithmus mit dem kleineren Einfluss-Faktors $X_u$ und den CTR der angemeldeten User die höchsten Relevanz-Bewertungen auf. Die aktuelle Springermedizin-Suche hat eine fast linear abfallende Form der Relevanz-Werte. Sehr ähnlich verhält sich die vierte Suchvariante des Reranking-Algorithmus, mit dem größeren Einfluss-Faktor für $Xu$ und der CTR aller User. Die schlechteste Konstellation der Suchvarianten des Reranking-Algorithmus, stellt die braune Linie mit dem kleiner Einfluss von $X_u$ und der CTR aller User dar. Diese fällt ab Position fünf stark ab.