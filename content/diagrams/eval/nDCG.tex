
\begin{figure}[H]
\centering 
\vspace{-1.5em}
\caption[nDCG-Auswertung]{nDCG-Auswertung aller Suchvarianten}
\label{fig:Evaluation:Auswertung:NDCGVarianten}

\footnotesize
\pgfplotstableread[col sep=semicolon]{content/diagrams/eval/nDCG.csv}\ndcg
\pgfsetplotmarksize{.5pt}

\begin{tikzpicture}
\begin{axis}[
	ybar,
    ymin=0,
    ymax=1.1,
	xtick=data,
	xticklabel style={align=center},
	xticklabels={{$\lambda$ für $X_u: 0.05$ \\ CTR aller \textit{angemeldeten User}}, {$\lambda$ für $X_u: 0.05$ \\ CTR \textit{aller User}}, {$\lambda$ für $X_u: 0.1$ \\ CTR aller \textit{angemeldeten User}}, {$\lambda$ für $X_u: 0.1$ \\ CTR \textit{aller User}}},
	width=11cm,
	height=4cm,
	scale only axis,
	ymajorgrids,
	yminorgrids,
	y label style={yshift=-4mm},
	x label style={yshift=-5mm},
    ylabel=\textbf{$nDCG_{10}$}, 
	xlabel=\textbf{Suchvarianten des Reranking-Algorithmus},
    legend style={font=\tiny},
   	legend pos=outer north east,
   	 legend style={cells={align=left}}
]
\addplot[nodes near coords, 
	every node near coord/.append style={xshift=-5pt,yshift=-1pt}, draw = blue,
       fill = blue!30!white] plot table [
    y=AVG_S,
    x=Index
] {\ndcg};
\label{pgfplots:plot1}
%\addlegendentry{$nDCG_{10}$ der aktuellen \\ Springermedizin-Suche}
\addplot[nodes near coords, 
	every node near coord/.append style={xshift=5pt,yshift=-1pt}, draw = red,
       fill=red!30!white] plot table [
    y=AVG_R,
    x=Index
] {\ndcg};
\label{pgfplots:plot2}
%\addlegendentry{$nDCG_{10}$ der Variante der Suche\\ mit dem Reranking-Algorithmus}
\end{axis}
\begin{axis}[
    ymin=0,
    ymax=1.1,
	xtick=\empty,
	ytick=\empty,
	width=11cm,
	height=4cm,
	scale only axis,
	ymajorgrids,
	yminorgrids,
    legend style={font=\tiny},
   	legend pos=outer north east
]
\addplot[draw=blue] plot table [
    y=AVG_S_Min,
    x=Index
] {\ndcg};
\label{pgfplots:plot3}
\addplot[draw=red] plot table [
    y=AVG_R_Min,
    x=Index
] {\ndcg};
\label{pgfplots:plot4}
\addplot[draw=blue] plot table [
    y=AVG_S_Max,
    x=Index
] {\ndcg};
\addplot[draw=red] plot table [
    y=AVG_R_Max,
    x=Index
] {\ndcg};
\end{axis}

%Legende
\matrix[
    matrix of nodes,
    anchor=north west,
    draw,% Rahmen um Legende
    inner sep=0.2em,
    column 1/.style={nodes={anchor=center}},
    column 2/.style={nodes={anchor=west},font=\tiny,cells={align=left}},
    draw
  ]    
at([xshift=0.25cm]current axis.north east){
    \ref{pgfplots:plot1}& $nDCG_{10}$ der aktuellen \\
    & Springermedizin-Suche\\
    & \ \\
    \ref{pgfplots:plot2}& $nDCG_{10}$ der Suchvariante \\ 
    & mit dem Reranking-Algorithmus\\
    & \ \\
    \ref{pgfplots:plot3}& Minimal- und Maximalwert des\\ 
    & $nDCG_{10}$ der aktuellen \\
    & Springermedizin-Suche\\
    & \ \\
    \ref{pgfplots:plot4}& Minimal- und Maximalwert des\\ 
    & $nDCG_{10}$ der Suchvariante \\ 
    & mit dem Reranking-Algorithmus\\};
\end{tikzpicture}


\vspace{-2.5em}
\end{figure}

Die oben dargestellte Abb. \ref{fig:Evaluation:Auswertung:NDCGVarianten} zeigt eine Mischform von einem Säulen- und Liniendiagramm. Die Säulen bezeichnen hierbei jeweils die verwendete Such-Implementierung. Die blauen Säulen beschreiben den durchschnittlichen $nDCG_{10}$ der aktuellen Springermedizin-Suche. Die Roten stellen jeweils den durchschnittlichen $nDCG_{10}$  einer Suchvariante der Suche mit dem Reranking-Algorithmus dar. Die Werte der x-Achse beschreiben hierbei welche dieser Suchvarianten jeweils mit der aktuellen Springermedizin-Suche verglichen wird. Der Faktor $\lambda$ für $X_u$ zeigt jeweils den Einfluss des Zufallswertes in den Reranking-Algorithmus und CTR beschreibt die Filterung der Click-Through-Daten anhand des Login-Status, bei der Berechnung der CTR. Die y-Achse beschreibt die Werteskala der nDCG-Auswertung. Diese bewegt sich zwischen 0 und 1. Die Linien stellen jeweils die Minimal- und Maximalwerte der gleichfarbigen Säulen dar.

\paragraph{Der Reranking-Algorithmus weist leichte Verbesserungen gegenüber der aktuellen Suche auf}
Wie wir sehen, liegen alle nDCG-Werte nahe beieinander. Die größten Verbesserungen weist der kleinere Einflussfaktor $\lambda$ für $X_u$~(0.05) des Zufallswertes, in Verbindung mit den Click-Through-Daten der angemeldeten User, auf. Die einzige Verschlechterung mittels Reranking-Algorithmus, tritt bei der Verwendung des höheren Einflussfaktors $\lambda$ für $X_u$ und der Berechnung der CTR aus den Click-Through-Daten aller User auf. Die Minimal- und Maximalwerte der beiden Suchen verhalten sich sehr ähnlich zu einander. Nur die letzte Konstellation, welche bereits beim durchschnittlichen nDCG-Wert Verschlechterungen zeigt, weist eine leichte Abweichung der Minimalwerte auf. Hingegen nähern sich alle Maximalwerte des nDCG, stark dem Maximalwert von 1 an. Anhand der Differenz der Minimalwerte der Suchvarianten können wir feststellen, dass der größere Einfluss-Faktor des Zufallswertes $\lambda$ für $X_u$, bessere Minimalwerte erzeugt.