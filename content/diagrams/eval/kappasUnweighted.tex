\begin{minipage}{0.4\linewidth}
In der Abb. rechts dargestellt, sehen wir dasselbe Liniendiagramm wie bereits in Abb. \ref{fig:Evaluation:Auswertung:Kappas}, nur dieses Mal mit dem ungewichteten Cohens Kappa Koeffizienten $K$. Wie wir hier sehen, verhalten sich die Linien der beiden Such-Implementierungen wieder sehr ähnlich und zeigen nur wenige starke Abweichungen voneinander. Allerdings hat sich das Muster der Linien und deren Werte auf der y-Achse stark verändert. 
\end{minipage}
\hfill
\begin{minipage}{0.55\linewidth}
\begin{figure}[H]
\centering 
\vspace{-2em}
\caption[Ungewichtete Kappa-Auswertung der Bewertungen]{Ungewichtete Kappa-Auswertung der Bewertungen} 
\label{fig:Evaluation:Auswertung:KappasUnweighted}

\footnotesize
\pgfplotstableread[col sep=semicolon]{content/diagrams/eval/kappasUnweighted.csv}\kappas
\pgfsetplotmarksize{.5pt}
  
\begin{tikzpicture}
\begin{axis}[
	width=8cm,
	height=4cm,
	scale only axis,
	xmajorgrids,
	xminorgrids,
    ylabel=\textbf{$K$}, 
	xlabel=\textbf{Suchterm-Analysen},
	ytick={{-0.8}, {-0.6}, {-0.4}, {-0.2}, 0, {0.2}, {0.4}, {0.6}},
	ymin={-0.8},
	ymax= {0.7},
    xtick={0, 20, 40, 60, 80},
    xmin=0,
    xmax=80,
    xlabel shift = 1in,
    ylabel shift = 1in,
    y label style={yshift=-4mm},
    legend style={font=\tiny},
    legend style={at={(axis cs:20,-0.75)},anchor=south west}
]
\addplot table [
    y=Normal,
    x=Index
] {\kappas};
\addplot table [
    y=Rerank,
    x=Index
] {\kappas};
\addplot table [
    y=Grenzwert,
    x=Index,
    no marks
] {\kappas};
\legend{aktuelle Springermedizin-Suche, Suche mit Reranking-Algorithmus, Grenzwert ($0.60$)}
\end{axis}
\end{tikzpicture}

\vspace{-1em}
\end{figure}
\end{minipage}

Die Werte für $K$ bewegen sich dieses Mal zwischen -0.8 und 0.6 und gehen damit teilweise in den negativen Bereich. Das Mindestmaß an Übereinstimmungsgüte wird nie erreicht. Nur eine Suchtermanalyse auf der aktuellen Springermedizin-Suche, nähert sich stark dem Grenzwert an. 