% !TEX root = ../thesis-example.tex
%
%************************************************
% Grundlagen
%************************************************
\chapter{Grundlagen}
\label{sec:Grundlagen}

\section{Grundbegriffe}
\label{sec:Grundlagen:Grundbegriffe}

\subsection{Semantik von User-Interaktionen}
\label{sec:Grundlagen:SemantikUserInteraktionen}

Nehmen wir als Beispiel die im vorherigen Abschnitt besprochene Suchanfrage \glqq chronische Dyspnoe\grqq{}. Würde eine andere Person stattdessen nach dem gleichbedeutenden Suchterm \glqq konstanter Atemnot\grqq{} suchen, können die Ergebnisse beider Suchresultate und somit auch deren Click-Trough-Daten gleichermaßen relevant sein. 

\subsection{Userrelevanz mittels Click-Trough-Rate (CTR)}
\label{sec:Grundlagen:UserrelevanzCTR}

\subsection{Result-Reranking mittels PBM Algorithmus}
\label{sec:Grundlagen:Result-RerankingPBM}

\section{Zusammenfassung}
\label{sec:Grundlagen:Zusammenfassung}