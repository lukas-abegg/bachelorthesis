% !TEX root = ../thesis-example.tex
%
%************************************************
% Grundlagen
%************************************************
\chapter{Grundlagen}
\label{sec:Grundlagen}

% Grundbegriffe
%************************************************

\section{Grundbegriffe}
\label{sec:Grundlagen:Grundbegriffe}

In diesem Kapitel werden wir die fachlichen Grundlagen zu unserem in Kapitel \ref{sec:Einfuehrung:Methodik} vorgestellten Lösungsansatz aufarbeiten. Wichtig hierfür ist das Verständnis für die Problemstellungen in der Interaktion zwischen den Nutzern des Suche und der Suche selbst. Dazu gehört die Auseinandersetzung mit dem Klick-Verhalten der User auf der Suche von Springermedizin. Mit der Click-Trough-Rate wollen wir eine Userrelevanz bestimmen. Dazu müssen wir die Click-Trough-Daten als Relevanzfeedback deuten können. Wie wir die Click-Trough-Rates berechnen, lernen wir in den Grundlagen zu unserem Reranking-Algorithmus. Diese Grundlagen sind notwendig für die Umsetzung des Lösungsansatzes in den nachfolgenden Kapiteln \ref{sec:Reranking} und \ref{sec:Implementierung}.

% Semantik von User-Interaktionen
%------------------------------------------------

\subsection{Semantik von User-Interaktionen}
\label{sec:Grundlagen:SemantikUserInteraktionen}

\subsubsection{Synonyme und verwandte Begriffe}
\label{sec:Grundlagen:SemantikUserInteraktionen:SynonymeVerwandte Begriffe}

Nehmen wir als Beispiel die Suchanfrage \glqq chronische Dyspnoe\grqq{}. Würde eine andere Person stattdessen nach dem gleichbedeutenden Suchterm \glqq konstante Atemnot\grqq{} suchen, könnten auf derer Bedeutungsgleichheit die Ergebnisse beider Suchresultate und somit auch deren Click-Trough-Daten gleichermaßen relevant sein. 

\subsubsection{Wenige Dokumente erhalten viele Klicks}
\label{sec:Grundlagen:SemantikUserInteraktionen:DocumentAttraction}


Um das Klick-Verhalten der User auf der Springermedizin-Suche zu verstehen, ist es wichtig anhand oft gesuchter Suchphrasen dieses Verhalten zu analysieren. Dazu wurde eine Analyse über einen Zeitraum von 30 Tagen erstellt und die zehn am häufigsten gesuchten Suchphrasen verwendet. Die Analyse vergleicht für jede Suchphrase die 20 Dokumente mit den meisten Klicks. Die Dokumente wurden hierbei nicht nach Position im Suchergebnis sondern nach Klick-Häufigkeit selektiert. Jeder Graph der folgenden Abbildung \ref{fig:Grundlagen:AnalyseKlicksTop10Suchergebnisse} stellt eine Suchphrase dar. Wie wir sehen, zeigen die meisten Graphen ein exponentiell stark abnehmendes Verhalten der Klick-Häufigkeiten. Dieses exponentielle Verhalten zeigt, dass einzelne Dokumente häufig und viele Dokumente selten bis nie angeklickt werden. Dieser Effekt kann wie in vielen natürlichen Phänomenen mit exponentiellem Verhalten, durch das Potenzgesetz (Power Law, siehe \cite{PowerLaw}) beschrieben werden. 

\begin{figure}[H]
\centering 
\vspace{-1em}
\caption[Analyse der 20 am häufigsten angeklickten Dokumente  der zehn meistgesuchten Suchphrasen. \textit{Zeitraum der Analyse: 19.08.16 - 19.09.16}]{Analyse der 20 am häufigsten angeklickten Dokumente  der zehn meistgesuchten Suchphrasen. \\ \textit{Zeitraum der Analyse: 19.08.16 - 19.09.16}}
\label{fig:Grundlagen:AnalyseKlicksTop10Suchergebnisse}
 
\pgfplotstableread[col sep=semicolon]{content/diagrams/clicks_top10_searchphrases_result.csv}\topSearchphrases
  
\begin{tikzpicture}
\begin{axis}[
	width=14cm,
	height=5cm,
	scale only axis,
	xmajorgrids,
	xminorgrids,
    ylabel=\textbf{Anzahl der Klicks}, 
	xlabel=\textbf{Dokumente sortiert nach Anzahl der Klicks},
    xtick=data,
    ymin=0,
    xmin=1,
    xmax=20,
    legend pos=north east,
    legend style={font=\tiny}
]
\addplot table [
    x=S1,
    x=Position
] {\topSearchphrases};
\addplot table [
    y=S2,
    x=Position
] {\topSearchphrases};
\addplot table [
     y=S3,
    x=Position
] {\topSearchphrases};
\addplot table [
     y=S4,
    x=Position
] {\topSearchphrases};
\addplot table [
     y=S5,
    x=Position
] {\topSearchphrases};
\addplot table [
     y=S6,
    x=Position
] {\topSearchphrases};
\addplot table [
     y=S7,
    x=Position
] {\topSearchphrases};
\addplot table [
     y=S8,
    x=Position
] {\topSearchphrases};
\addplot table [
     y=S9,
    x=Position
] {\topSearchphrases};
\addplot table [
     y=S10,
    x=Position
] {\topSearchphrases};
\legend{borreliose ($1015$ Suchergebnisse), copd ($17337$ Suchergebnisse), dgrm-jahrestagung $2016$ ($18$ Suchergebnisse), diabetes ($148755$ Suchergebnisse), dyspnoe ($10601$ Suchergebnisse), forensische traumatologie ($150$ Suchergebnisse), gicht ($1188$ Suchergebnisse), hypertonie ($12765$ Suchergebnisse), mmw ($12666$ Suchergebnisse), vorhofflimmern ($4981$ Suchergebnisse)}
\end{axis}
\end{tikzpicture}

\vspace{-2em}
\end{figure}

Betrachten wir die Graphen, können wir vor allem für die ersten fünf analysierten Dokumente verglichen mit den restlichen analysierten Dokumenten, hohe Klick-Häufigkeiten feststellen. Daraus lässt sich die Vermutung ableiten, dass einzelne Dokumente eine sehr hohe Relevanz für die entsprechende Suchanfrage aufweisen und nur wenige Dokumente auf die User als relevant wirken. Ein weitere Vermutung ist, dass der zu durchsuchende Content wenig relevante Dokumente hat. Die Suchphrasen lassen auf sehr diverse Suchintentionen deuten. Es handelt sich hierbei unter anderem um Krankheiten, Zeitschriften und Behandlungen mit mehreren tausend Suchergebnissen. Die Wahrscheinlichkeit, dass wenig relevanter Content für die meisten der analysierten Suchphrasen zutrifft, sollte aufgrund der hohen Anzahl an gefundenen Suchergebnissen zu diesen Suchphrasen, relativ gering sein. Wir müssen darum eher davon ausgehen, dass sich die User auf einzelne im Suchresultat weit oben stehende Dokumente festfahren. Das könnte an schlechten Suchergebnissen und somit an einer schlechten Suchqualität liegen. Um jedoch ein genaueres Bild über das Verhalten erstellen zu können müssen wir einen Vergleich mit der nachfolgenden Analyse in Abbildung \ref{fig:Grundlage:AnalyseKlicksPositionen} ziehen.

\subsubsection{Niedrige Positionen werden häufiger angeklickt}
\label{sec:Grundlagen:SemantikUserInteraktionen:RankExamination}


In der unten folgenden Analyse sehen wir das positionsbezogene Klick-Verhalten der User auf der Springermedizin-Suche. Dazu wurden über den Zeitraum von einem Monat, die letzten 1000 Suchanfragen ausgewertet. Dargestellt sehen wir die Häufigkeitsverteilung der Klicks als Graph. Wir beschränken uns hierbei auf die ersten 20 Positionen der Suchresultate. Wie wir sehen, nimmt die Anzahl der Klicks mit zunehmender Position exponentiell ab. Dieser Effekt kann ebenfalls, wie in Abb. \ref{fig:Grundlagen:AnalyseKlicksTop10Suchergebnisse}, durch das Potenzgesetz (Power Law, siehe \cite{PowerLaw}) beschrieben werden. 

\begin{figure}[H]
\centering 
\vspace{-1em}
\caption[Analyse der Klicks auf die ersten 20 Positionen der Suchergebnisse aller Suchanfragen. \textit{Zeitraum der Analyse: 19.08.16 - 19.09.16}]{Analyse der Klicks auf die ersten 20 Positionen der Suchergebnisse aller Suchanfragen. \\ \textit{Zeitraum der Analyse: 19.08.16 - 19.09.16}}
\label{fig:Grundlage:AnalyseKlicksPositionen}

\footnotesize
\pgfplotstableread[col sep=semicolon]{content/diagrams/clicks_top1000_ranks_result.csv}\topRanks
  
\begin{tikzpicture}
\begin{axis}[
	width=14cm,
	height=3cm,
	scale only axis,
	xmajorgrids,
	xminorgrids,
    ylabel=\textbf{Anzahl der Klicks}, 
	xlabel=\textbf{Position im Suchergebnis},
	nodes near coords, 
	 every node near coord/.append style={xshift=+10pt,yshift=-1pt},
    xtick=data,
    ymin=0,
    xmin=1,
    xmax=20,
    legend style={font=\tiny}
]
\addplot table [
    x=Position,
    y=Klicks
] {\topRanks};
\legend{Anzahl Klicks}
\end{axis}
\end{tikzpicture}

\vspace{-2em}
\end{figure}

Betrachten wir den Graphen, sehen wir, dass besonders die erste Position, auffällig oft angeklickt wird. Daraus könnten wir die Vermutungen ableiten, dass die Suche eine sehr gute Qualität besitzt, weil die zu oberst angezeigten Dokumente, sehr relevant sind und die meisten User der Suchmaschine vertrauen. Wie wir aus den Analysen von \cite{Joachims} lesen können, müssen wir davon ausgehen, dass die Häufigkeit des Klicks auf die ersten Positionen des Suchresultates eher dem Vertrauen der User der Suchmaschine, als der Qualität der Suche geschuldet ist. Vergleichen wir die Analyse aus Abb. \ref{fig:Grundlagen:AnalyseKlicksTop10Suchergebnisse} mit dieser Analyse, sehen wir ein sehr ähnliches Muster in der Häufigkeitsverteilung der Klicks. Wir können anhand der Klick-Zahlen ebenfalls vermuten, dass die am häufigsten angeklickten Dokumente, sich dabei auf den ersten Positionen des Suchergebnisses befunden haben.

% Userrelevanz mittels Click-Trough-Rate (CTR)
%------------------------------------------------

\subsection{Userrelevanz mittels Click-Trough-Rate (CTR)}
\label{sec:Grundlagen:Click-Trough-Daten}

Um mit Click-Trough-Daten arbeiten zu können, müssen wir zuerst verstehen, was Click-Trough-Daten sind und wie sie entstehen. 

\subsubsection{Was sind Click-Trough-Daten und wie entstehen diese?}
\label{sec:Grundlagen:Grundbegriffe:Click-Trough-Daten:WasSindClick-Trough-Daten}

Click-Trough-Daten sind Tracking-Daten. Tracking-Daten entstehen durch die Interaktion zwischen dem User der Applikation und der Applikation selbst. Sie verfolgen das Verhalten der User auf der Applikation und speichern diese in einer Datenbank, in unserem Fall in Webtrekk ab. Die für uns interessanten Tracking-Daten entstehen, wenn der User auf der Suche von Springermedizin ein Anfrage stellt und darauf folgend, ein Element aus dem Suchresultat anklickt.

\subsubsection{Wie werden die Click-Trough-Daten in Webtrekk gespeichert?}
\label{sec:Grundlagen:Grundbegriffe:Click-Trough-Daten:SpeichernClick-Trough-Daten}

Die Speicherung der Daten auf Webtrekk übernimmt die Springermedizin-Applikation. Führt ein User eine Suche durch und klickt dabei ein Resultat an, sendet die Springermedizin-Applikation die Tracking-Informationen an Webtrekk. Die Tracking-Daten für diese Aktion, setzen sich zusammen aus der Suchanfrage, dem Zeitpunkt der Suche, den Userdaten, der angeklickten Position im Suchresultat und den Dokumentinformationen zum angeklickten Dokument.

\subsubsection{Wie können wir Click-Trough-Daten aus Webtrekk lesen?}
\label{sec:Grundlagen:Grundbegriffe:Click-Trough-Daten:LesenClick-Trough-Daten}

Webtrekk ist ein Analysetool. Das heißt für uns, wir können nicht direkt auf die Datenbank mit den Tracking-Daten zugreifen. Um die Tracking-Daten lesen zu können, müssen wir eine Analyse auf Webtrekk ausführen. Mithilfe dieser Analyse können wir uns die Click-Trough-Daten so zusammenstellen lassen, wie wir sie für die Berechnung der Click-Trough-Rate benötigen.
\\
\\
Die Click-Trough-Daten bestehen aus einzelnen \textit{Klick-Werten}. Ein Klick-Wert beschreibt die Anzahl der Klicks, die zu einer bestimmten Suchanfrage auf ein bestimmtes Dokument gemacht wurden und auf welcher Position im Suchresultat sich dieses Dokument dabei befunden hat. Die Webtrekk-Analysen geben uns eine Sammlung von Klick-Werten zurück. Wir können bei diesen Analysen die Klick-Werte nach Suchbegriffen oder auch Suchtermen filtern und den Zeitraum mitgeben, in welchen die Suchanfragen durchgeführt wurden. Des weiteren gibt es die Möglichkeit, weitere Filter wie die Anzahl zurückzugebender Klick-Werte oder auch den \glqq Login-Status\footnote{Mit Login-Status wird zwischen einem zum Zeitpunkt der Suche auf der Springermedizin-Applikation angemeldeten und nicht angemeldeten User unterschieden} des Users\grqq{} zu setzen. 

\subsubsection{Wie sehen die Click-Trough-Daten aus?}
\label{sec:Grundlagen:Grundbegriffe:Click-Trough-Daten:AussehenClick-Trough-Daten}

Eine Beispiel für einen Klick-Wert wie er von einer Webtrekk-Analyse ausgespielt wird, sieht wie folgt aus:

\begin{tabular}{|p{0.8\textwidth}|p{0.15\textwidth}|}\hline
	\textbf{Click-Trough-Daten} & \textbf{Klick-Wert} \\ \hline
	searchresult-1.Course.chronische Dyspnoe bei Erwachsenen.10621768.chronische Dyspnoe & 5 \\ \hline
 \end{tabular}

Hier die Aufschlüsselung der Click-Trough-Daten:

\begin{tabular}{|p{0.15\textwidth}|p{0.15\textwidth}|p{0.27\textwidth}|p{0.1\textwidth}|p{0.2\textwidth}|}\hline
	\textbf{Position} & \textbf{Dokumenttyp} & \textbf{Titel} & \textbf{ID} & \textbf{Suchterm} \\ \hline
	searchresult-1 & Course & chronische Dyspnoe bei Erwachsenen & 10621768 & chronische Dyspnoe \\ \hline
 \end{tabular}
 
Die Click-Trough-Daten lassen sich wie folgt lesen. In diesem Beispiel haben die User mit der Suchanfrage \glqq chronische Dyspnoe\footnote{Als Dyspnoe wird eine unangenehm erschwerte Atemtätigkeit bezeichnet}\grqq{} gesucht. Dabei haben sie das Dokument mit der ID 10621768 angeklickt. Dieses hat sich dabei auf der Position eins der Suchresultate befunden. Es wurde insgesamt fünfmal angeklickt in der gesuchten Periode. 


\subsubsection{Das Userverhalten bestimmt über die Relevanz eines Klicks}
\label{sec:Reranking:Grundlagen:UserrelevanzCTR:Aussagekraft}

Zur Definition der Click-Trough-Daten gibt es verschiedene Ansätze. Wie in \cite{Joachims} beschrieben, können anhand des Verhaltens der User vor während und nach dem Klick Schlussfolgerungen für die Relevanz des Klicks gemacht werden.
\\
\\
Wir bauen dabei auf den Erkenntnissen aus \cite{Joachims} aus. In diesen Analysen wurde untersucht, wie aussagekräftig ein Klick während eines Suchvorgangs ist.  


% Result-Reranking mittels PBM Algorithmus
%------------------------------------------------

\subsection{Result-Reranking mittels PBM Algorithmus}
\label{sec:Grundlagen:Result-RerankingPBM}

% Zusammenfassung
%------------------------------------------------

\section{Zusammenfassung}
\label{sec:Grundlagen:Zusammenfassung}