% !TEX root = ../thesis-example.tex
%
\chapter{Fazit}
\label{sec:Fazit}

\section{Eingesetzte Fähigkeiten und Kenntnisse aus dem Studium}
\label{sec:Fazit:SP}

Da das Praktikum sehr programmierlastig aufgebaut war, kamen mir vor allem in der alltäglichen Tätigkeiten die Erfahrungen im Bereich Programmierung 1 und 2 zu Gute. Zwar ist der in Scala verfolgte Ansatz nicht ganz mit dem aus Java zu vergleichen, aber die Grundkenntnisse zu Programmstrukturen, Sichtbarkeit von Methoden und Attributen, dem richtigen Programmierstil und der Objektorientierten Programmierung haben sicherlich dazu beigetragen, schnell sich in eine neue Sprache einzuarbeiten. Ich denke eine wichtige Erkenntnis hierbei ist auch, dass zwar jede Programmiersprache einen eigenen Syntax besitzt, aber die Grundlagen der Programmierung überall eingesetzt werden können. Ein weiterer wichtiger Punkt ist das Arbeiten mit der MVC-Pattern. Dieser Ansatz wird auch in Play sowie vielen anderen Frameworks verfolgt. Versteht man diese Architektur, fällt es einem leicht, die Funktionsweise solcher Frameworks nachzuvollziehen. Um eine Applikation aufzubauen, werden aber noch mehr Entwurf-Muster als MVC benötigt. Hier kommt das Wissen aus FPA zum Einsatz. Mithilfe dieses Konzeptwissens über Pattern wie Fabrikmethoden, Singleton oder auch Module, können komplexere Programmierkonstrukte aufgebaut werden um beispielsweise eine Persistenzschicht zu einer Datenbank zu erstellen. 
\\
\newline
Aus Sicht der Frontend-Entwicklung war es sicherlich hilfreich sich in MME 1 mit HTML 5 und CSS 3 ausgetobt zu haben. Nebst dem Verständnis für den richtigen Einsatz von CSS-Klassen und HTML-Strukturen, konnte so auch der sinnvolle Einsatz von Style-Frameworks wie Bootstrap analysiert werden. Im Vergleich zu MME 1, wo nur mit solchen Style-Frameworks gearbeitet wurde, setzt Springer eine von Hand programmierte Lösung. Die wichtigste Erkenntnis hierbei ist wohl, das zwar mit Style-Frameworks wie Bootstrap schnell gute Ergebnisse erzielt werden können, aber für den Einsatz in einer großen Plattform, solche Frameworks zu hinterfragen sind, da diese viele und oft auch zu viele Funktionen anbieten und die Applikation damit nur unnötig vergrößern. Ausserdem ist der Aufwand für eine handgemachte Lösung schlussendlich nicht umbedingt größer, da die Anpassung des Stils auf den zu der Firma passenden Stil (CI), bei einem Style-Framework wie Bootstrap ebenfalls eher grösser werden kann.
\\
\\
Zum Verständnis der Web-Architektur, war es vor allem hilfreich in MME 2 sich mit der Backend-Entwicklung eines NodeJs-Servers auseinandergesetzt zu haben. In dem die Serverseite mit einer REST-Schnittstellen entwickelt und gleichzeitig dazu die Clientseite mithilfe einer kleinen NodeJs-Applikation angebunden wurde, konnten die kompletten Kommunikationswege zwischen Client- und Serverseite nachvollzogen werden. Auf dieser Basis ist es dann auch leichter, mächtigere Frameworks, wie in diesem Fall Play nachzuvollziehen, da das Prinzip im Grundsatz dasselbe ist. Überlegt man sich nun, wie die Web-Applikation über das Internet erreichbar wird, erinnert man sich an Verteilte System. Dort hat man gelernt, dass mithilfe einer URL über die DNS-Auflösung der Weg zur IP des Servers der Web-Applikation hergestellt werden kann. Damit die Web-Applikation auch im Internet erreichbar wird, muss diese in einer DMZ-Zone liegen. Denkt man nun noch darüber nach, dass mehrere Personen gleichzeitig Anfragen gegen diese Web-Applikation stellen können, wird ein Loadbalancer davorgesetzt und zur Sicherheit ein SSL-Zertifikat hinzugefügt. Wird nun der Weg aufgezeichnet, erhält man schon fast das komplette Architekturkonstrukt, um eine Verbindung zwischen Client und Server herzustellen. 
\\
\\
Die Entwicklung eines Software-Produktes verlangt aber auch eine gute Projektplanung und Konzeption. Auch als Programmierer ist mittlerweile ein wenig Scrum-Wissen von Vorteil. Ohne dies funktioniert in einem sauberen Webprojekten nichts. In Software-Engineering 1 + 2, sowie in Qualitäts- und Projektmanagement hat man einen Grundeinblick in alle Phassen eines solchen Projektes gekriegt. Zwar unterscheidt sich das Wasserfall-Prinzip doch ein wenig vom moderneren Scrum Prinzip, welches bei Springer eingesetzt wird, doch für das Grundverständnis reicht dies. 

\section{Herausforderungen und Schwierigkeiten}
\label{sec:Fazit:HS}

Aus fachlicher Sicht, war der Technologiestack sicherlich die größte Herausforderung. Nur um nur die Core-Applikation (Nemo-Core) zu verstehen, müssen 3 verschiedene Programmier-/Skript-Sprachen (Scala, SBT und Javascript), das mächtige Play-Framework und einige Zusatzsoftware wie beispielsweise Silouhette für die Authentifizierung und unzählige eingesetzte Libraries verstanden werden. Dazu kommt die Solr zur Suche und der Auslieferung von Content, sowie einem eLearning-System auf AngularJs-Basis und die komplette Deploy-Pipeline. Die Übersicht über all diese Applikationen in 3 Monaten zu gewinnen, ist ein Ding der Unmöglichkeit. Selbst in meinen bisherigen 2 Jahren, habe ich noch immer nicht alle Applikationen und Services beim Projekt Nemo gesehen. Der komplette Springer-Konzern hat diesen Stack nochmals in einigen verschiedenen Variationen. Dies zeigt aber auch mit welcher Herausforderung ein Entwickler in einem solchen Projekt konfrontiert wird. Ein Fullstack-Entwickler im klassischen Sinne ist mit diesen Anforderungen gar nicht mehr möglich. Das schöne hierbei ist jedoch, dass man immer wieder neue Themen und Technologien anschauen und anwenden kann.
\\
\\
Zum Technologiestack kommt zusätzlich die größe des Projektes hinzu. Ist man das erste Mal mit einem Projekt dieser größe konfrontiert, ist es schwierig die Übersicht zu behalten. Man lernt zwar die meisten Personen und den Aufbau der Projektorganisation mit der Zeit kennen, doch in 3 Monaten wäre dies definitiv nicht möglich. 
\\
\\
Ein weiterer Punkt sind die zu beachtenden Faktoren und Prozesse beim Programmieren. Man muss sich Guidelines verinnerlichen, sowie den im Projekt definierten Prozess für die Entwicklung eines Produktes beachten. Dazu zählen Dinge wie, "wie wird ein Git-Branch benannt" über "wie werden die Tests richtig geschrieben" bis hin zu "wie wird ein Pull Request geschrieben". All diese Dinge sind zwar nicht schwierig, müssen aber alle gut und sauber abgeklärt werden. 
\\
\\
Die schönste aber auch herausfordernste Thema ist Scala. Hierbei war die Herausforderung nicht das Programmieren von paar funktionierenden Zeilen Code. Die Herausforderung war das sogenannt strukturiert schöne und perfomante Programmieren. Wann setzt man Future()'s ein? Wo   trenne ich eine Funktion in mehrere Zeilen Code und wo kann ich alle Funktionen hintereinader aufrufen? Das hat viel mit Erfahrung zu tun. Der beste Code ist eine Mischung aus performant, schön strukturiert und gut lesbar. Man entwickelt schliesslich nicht für sich alleine sondern so, dass dieses Coding später weiterverwendet werden kann. 

\section{Bewertung des Praktikums insgesamt}
\label{sec:Fazit:BP}

\subsection{Arbeitsklima und Arbeitskultur}
\label{sec:Fazit:BP:AA}

Ein wichtiger Punkt im Projektumfeld ist das Arbeitsklima. Bei Springer wird versucht mit flachen Hierarchien zu arbeiten. Jeder soll und muss sich einbringen und seinen Beitrag zum Ganzen leisten. Bei Diskussionen sind Argumente wichtig und die Lösungsvorschläge mit den besten Argumenten gewinnen. Dadurch lernt man gut zu argumentieren. Dabei ist wichtig nicht nur wie, sondern auch warum eine Lösung gut und sinnvoll ist, in die Argumente einfließen zu lassen. 
\\
\\
Um Probleme und Schwierigkeiten anzusprechen, aber natürlich auch für Neuigkeiten aus dem Projektumfeld, werden einmal im Monat Teammeetings durchgeführt. Hier wird eine offene Kommunikationskultur gepflegt. Für persönliche Anliegen und um Feedback einzuholen sind die One{\&}One's angedacht.
\\
\\
Ein weiteres interessantes Meeting sind die Tech-Talks. Bei diesen werden neu umgesetzte Features und offene Fragen zur Architektur beantwortet. Das Ziel ist hierbei nicht, dass jemand die ganze Zeit vorne steht und Fragen beantwortet, sondern dass gemeinsam Code analysiert wird, Wissen eingebracht und sich so zum Schluss allen die Lösung erschliesst.
\\
\\
Es ist keine Startup-Kultur sondern eine gefestigte Firma wo Arbeitszeiten und Freizeit neben der Arbeit ein wichtiges Gut sind. Die Arbeitskultur an sich ist aber relativ locker und dennoch konzentriert. Jeder arbeitet für sich oder in kleinen Teams. Die Teams sind in Grossraumbüros unterteilt. Dadurch sitzen die Leute die am meisten miteinander zu tun haben nah beieinander. Hängt man an einem Problem fest, ist es aber kein Problem, sich mit einem anderen Entwickler zusammen zu sitzen und diesen um Rat zu fragen. Der zentrale Punkt für lockere Diskussionen ist dennoch die Küche mit der Kaffeemaschine. Diese spielt eine doch sehr zentrale Rolle für das Wohlbefinden und die gute Laune der Leute. 
\\
\\
Im Großen und Ganzen wird aber viel versucht, in gemeinsamen Meetings alle Mitarbeiter oder auch Teams zusammen zu bringen, um soziale Interaktionen zu fördern. Im Projekt und den Teams wird man gut integriert und lernt viel. Die eigene Meinung wird als wichtig betrachtet und es ist egal ob ein Entwickler Student oder 10 Jahre Erfahrung hat, gute Argumente zählen.

\subsection{Fachliche Sicht}
\label{sec:Fazit:BP:FS}

Aus fachlicher Sicht beurteilt, konnte ich auf jedenfall meine Scala-Kenntnisse weiter schulen und habe auch im Bereich asynchrone Programmierung etwas dazugelernt. Interessant war auch der Aufbau eines kleinen eigenen Projektes in einem Zweierteam. Der Prozess-Ablauf vom Startpunkt, wo die Aufgabenstellung analysiert und ein Lösungsweg evaluiert wird. Über die Zwischenstandspräsentation, wo selbständig Meetings organisiert und genau vorbereitet wird, wie der Sinn und Zweck des Produktes den Zuhörern richtig verkauft wird. Bis hin zum Ende, wo die Live-Setzung des Projektes und Übergabe an die anderen Entwickler stattfindet. 
\\
\\
Zudem habe ich jetzt nochmals einiges zum Thema Projektmanagement und Organisation hinzugelernt. Ich weiss mittlerweile wie eine neue Aufgabenstellung sauber und korrekt im Projektumfeld umgesetzt wird, wie die korrekte Vorgehensweise ist, auf welche Dinge während der Umsetzung geachtet werden muss und wann die Qualität des Ergebnisses gut ist. 

\subsection{Persönliche Sicht}
\label{sec:Fazit:BP:PS}

Eine wichtige Erfahrung für mich für zukünftige Projekte und arbeiten ist das Einfordern von Feedback und saubere Abklären der Anforderungen mit allen involvierten Parteien. Im Studium ist man sich gewöhnt, alle Informationen zu kriegen, ohne selbst etwas einfordern zu müssen. Im Arbeitsalltag ist dem nicht so. Will man neue Dinge ausprobieren oder ein Feedback zur eigenen Arbeit kriegen, muss dies eingefordert werden. Sonst geht man unter. Beim Abklären von Anforderungen ist es wichtig, präzise Fragen zu stellen um den Befragten im Gespräch dahin zu führen, wo er dir genau die Informationen erzählt, die man benötigt. 
\\
\\
Eine sehr interessante Festellung die ich erst während des Praktikums gemacht habe, war der Einfluss die Präsenz vor Ort auf die Wahrnehmung des Projektalltags und die Leute. Mehr Präsenz bedeutet mehr Einbeziehung in den Projektalltag. In meinem Fall ist mir stark aufgefallen, dass seit ich nicht nur einen Teil der Woche vor Ort bin, sondern die komplette Woche, der Projektalltag und die vielen Meetings mir mehr bewusst geworden sind. Vorher lief das alles so nebenbei. Ausserdem ist mir aufgefallen, dass dadurch dass ich mittlerweile einige kleinere Dinge wie Deploy-Automatisierungen oder Go-Live-Planungen selbständig durchgeführt habe, die Leute mich öfter um Hilfe und Rat fragen. Ein weiterer wichtiger Punkt ist die Selbständigkeit. Mit guten und selbständig durchgeführten Arbeiten steigert sich das Vertrauen der Mitarbeiter in einen. Man kriegt herausforderndere und größere Arbeiten, bis hin zu komplexen Entwicklungsarbeiten.

\subsection{Gesamtfazit}
\label{sec:Fazit:BP:GFz}

Das Praktikum war interessant und hat mich ein gutes Stück weiter gebracht. Mit dem Projektumfeld und dem Technologiestack bei Springer habe ich einen sehr guten Eindruck in die profesionelle Webentwicklung erhalten können und auch persönlich mit selbst weiterentwickeln können. Es sind auf jedenfall andere Eindrücke, als die, welche in einem Startup gemacht werden werden. Aber ich denke als Start in die Berufslaufbahn als Entwickler ist dieses Umfeld bei Springer auf jedenfall sinnvoller als ein kleines Startup, da hier ein professionelles Umfeld kennengelern werden kann und ein großer Technologiestack zur Verfügung steht.